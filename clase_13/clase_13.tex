\documentclass[11pt, aspectratio=169, xcolor=table,hyphens]{beamer}
\usepackage[utf8]{inputenc}
\usepackage[T1]{fontenc}
\usepackage{lmodern}
\usepackage[spanish]{babel}
\usepackage{pdfrender}
\usepackage{tikz}
\usepackage{graphicx}
\usepackage{xcolor}
\usepackage{hyperref}
\usepackage{cite} %permite salto de línea en referencias largas
\usepackage{ragged2e}
\usepackage{smartdiagram}
\bibliographystyle{../apalike-es} %archivo de estilo apa ../apa-es en español en el directorio
\usepackage{pgfplots}
\usepackage{pgf-pie}
\usepackage[]{wasysym}
\usepackage{tabularx} % para ajuste del tamaño de la tabla
\usepackage{multirow} % juntar filas en una tabla w
\usepackage{chemfig}
\usepackage{chemmacros}

\newcommand\encircle[1]{% Poder crear letras dentro de círculos
	\tikz[baseline=(X.base)]
	\node (X) [draw, shape=circle, inner sep=0] {\strut #1};}

\usepackage{../template/uandes169template}

%%%%%%%%%%%%%%%%%%%%%%%%%%%%%% INICIO DEL CONTENIDO %%%%%%%%%%%%%%%%%%%%%%%%%%%%%%%%%%%%%%%%%%%%%%

\begin{document}
\author{Prof. Daniel Muñoz \\
	\href{mailto:dmunoz@miuandes.cl}{\texttt{dmunoz@miuandes.cl}}}
\title{Retroalimentación efectiva}
\subtitle{Clase 13}
\date{28 de mayo}
\maketitle

\section{Informa Prueba 2}

\begin{frame}[allowdisplaybreaks]
	\frametitle{Informe Prueba 2}
	\begin{itemize}
		\item[OA1.] (89\%) Crear objetivos de aprendizaje coherentes con la taxonomía de Anderson y Krathwohl.
		\item[OA2.] (90\%) Crear indicadores de evaluación a partir de objetivos de aprendizaje utilizando la taxonomía de Anderson y Krathwohl.
		\item[OA3.] (91\%) Aplicar la tabla de especificaciones para planificar la evaluación.
	\end{itemize}
\end{frame}

\section{Evaluación Formativa}

\begin{frame}{La evaluación formativa}
	\begin{columns}
		\begin{column}{.5\linewidth}
			\begin{itemize}[<+->]
				\item Recuerde el ciclo de la evaluación
				\item La evaluación formativa también conocida como \textit{evaluación como aprendizaje}.
				\item Pero no sirve de nada si no genera aprendizajes, entonces...
				\item Para eso nos basaremos en el trabajo de Margaret Heritage
			\end{itemize}
		\end{column}
		\begin{column}{.5\linewidth}
			\begin{figure}
				\includegraphics<1>[width=.8\linewidth]{../imagenes/assesment_process}
				\includegraphics<2>[width=.8\linewidth]{../imagenes/gold_coin}
				\includegraphics<3>[width=.8\linewidth]{../imagenes/question}
				\includegraphics<4>[width=.8\linewidth]{../imagenes/heritage}
				\caption{
					\only<1>{El ciclo de la evaluación}
					\only<2>{Es la moneda de oro de la evaluación}
					\only<3>{¿Cómo generamos aprendizajes?}
					\only<4>{Margaret Heritage, profesora inglesa experta en evaluación formativa}
				}
			\end{figure}
		\end{column}
	\end{columns}
\end{frame}

\begin{frame}{El ciclo de la evaluación formativa. \cite{AC2010}}
	\begin{columns}
		\begin{column}{.5\linewidth}
			\begin{itemize}
				\item<1-> El ciclo de la evaluación formativa.
				\item<2-> Consiste de tres etapas:
				      \begin{enumerate}
					      \item<3-> ¿Hacia donde voy?
					      \item<4-> ¿Dónde estoy?
					      \item<5-> ¿Cómo sigo?
				      \end{enumerate}
				\item<6-> En esto la \textit{retroalimentación} es fundamental ya que nos permite superar esa \textit{brecha}.
			\end{itemize}
		\end{column}
		\begin{column}{.5\linewidth}
			\only<1-2>{\includegraphics[width=\linewidth]{../imagenes/ciclo_formative}}
			\only<3>{Los objetivos de aprendizaje, criterios de logro deben ser de explicito conocimiento y entendimiento. El estudiante debe saber \textit{Para qué vino a clases}}\\
			\only<4>{El estudiante debe saber cual es su nivel, dónde está en al ruta del aprendizaje, para eso evaluamos y le señalamos exactamente cómo está el aprendizaje logrado hasta ahora  y ...}
			\only<5>{Cómo mejorar para llegar al destino}
			\only<6>{
				\begin{tikzpicture}[node distance= 3cm]
					\node[rectangle, rounded corners, minimum width=3cm, minimum height=1cm, text centered, draw=black] (zda) {Estado actual};
					\node[rectangle, rounded corners, minimum width=3cm, minimum height=1cm, text centered, draw=black, below of = zda] (zdp) {Aprendizaje};
					\draw[|-|,thick,>=stealth] ([xshift = 2cm] zda.south)  -- node[anchor=west] {Brecha} ([xshift = 2cm] zdp.north);
					\draw[>-<,thick,>=stealth] (zda) -- node[anchor=east] {Retroalimentación} (zdp);
				\end{tikzpicture}}
		\end{column}
	\end{columns}
\end{frame}

\section{La retroalimentación efectiva}

\begin{frame}{La retroalimentación efectiva}
	\begin{columns}
		\begin{column}{.5\linewidth}
			\begin{itemize}[<+->]
				\item Primero que todo, ¿vale la pena retroalimentar?
				\item ¿En qué consiste una retroalimentación?
				\item Una correcta retroalimentación debe responder a criterios de calidad.
			\end{itemize}
		\end{column}
		\begin{column}{.5\linewidth}
			\begin{figure}
				\includegraphics<1>[width=\linewidth]{../imagenes/feedback_hattie}
				\only<2>{
					\begin{columns}
						\begin{column}{.3\linewidth}
							\begin{block}{}
								\flushright
								Lo que el estudiante hizo bien
							\end{block}
						\end{column}
						\begin{column}{.3\linewidth}
							\centering +
						\end{column}
						\begin{column}{.3\linewidth}
							\begin{block}{}
								\flushleft Como mejorar
							\end{block}
						\end{column}
					\end{columns}
				}
				\only<1>{\caption{``Hattie Rank''  \cite{HATTIE2017}} }
				\only<2>{\caption{Estructura de una \textit{retroalimentación efectiva}}}
			\end{figure}
			\only<3->{
				Una \textit{retroalimentación efectiva} debe:
				\begin{itemize}
					\item<4-> estar relacionada con la meta.
					\item<5-> entregar pistas, \textbf{no soluciones}.
					\item<6-> captar el razonamiento del estudiante.
					\item<7-> ser usada por el estudiante.
					\item<8-> enfocarse en el aprendizaje y \textbf{no} el individuo.
				\end{itemize}
			}
		\end{column}
	\end{columns}
\end{frame}

\section{Otros modelos de retroalimentación}

\begin{frame}[shrink, squeeze]
	\frametitle{Modelo de tres niveles \cite{hattie07}}
	\begin{columns}
		\begin{column}{.5\textwidth}
			\scriptsize
			El modelo de Hattie y Timperley (2007) es idéntico al de Heritage, pero ellos distinguen tres niveles de feedback:

			\begin{itemize}[<+->]
				\scriptsize
				\justifying
				\item Nivel 1. Tarea. En este nivel usted retroalimenta las tareas del estudiante. Aquí usted se centra en el ``qué''
				\item Nivel 2. Proceso. En este nivel usted retroalimenta el proceso del estudiante. Aquí usted se centra en el ``cómo''
				\item Nivel 3. Autoregulación. En este nivel usted retroalimenta la capacidad del estudiante para monitorearse. Aquí usted desarrolla la metacognición del estudiante.
				\item Nivel 4. Yo. Este nivel es cuando se retrolalimenta a la persona, pero al igual que Heritage, es la retroalimentación menos efectiva y se recomienda no usarla.
			\end{itemize}
		\end{column}

		\begin{column}{.5\textwidth}
			\only<1>{
				\begin{itemize}
					\item ``Esta suma está bien calculada.''

					\item ``Aquí hay un error de ortografía en la palabra 'conciencia'.''

					\item ``Te falta desarrollar más el segundo punto de tu ensayo.''

					\item ``Tu gráfico es claro, pero los ejes no están etiquetados correctamente.''

				\end{itemize}
			}
			\only<2>{
				\begin{itemize}
					\item ``La forma en que organizaste tus ideas en este esquema te ayudó a estructurar bien el ensayo.''

					\item ``Necesitas revisar los pasos de la división antes de hacer la multiplicación.''

					\item ``Intenta utilizar un organizador gráfico para comparar y contrastar las ideas antes de escribir.''

					\item ``¿Qué hiciste para investigar la información? Considera usar fuentes académicas para tus próximas búsquedas.''
				\end{itemize}
			}
			\only<3>{
				\begin{itemize}
					\small
					\item ``¿Qué aprendiste de los errores que cometiste en esta tarea? ¿Cómo aplicarás eso en la siguiente?''

					\item ``Antes de entregar, ¿cómo revisaste tu trabajo para asegurarte de que cumplía con todos los criterios?''

					\item ``¿Qué pasos tomaste cuando te sentiste atascado en este problema? ¿Podrías intentar una estrategia diferente si vuelve a ocurrir?''

					\item ``Me doy cuenta de que has hecho un buen trabajo al planificar tus tiempos. ¿Cómo te ayudó eso a completar la tarea a tiempo?''
				\end{itemize}
			}
			\only<4>{
				\begin{itemize}
					\item ``¡Eres un genio!''

					\item ``Qué bueno, hiciste un trabajo excelente.''

					\item ``Siempre eres tan creativo.''
				\end{itemize}
			}
		\end{column}
	\end{columns}

\end{frame}

\begin{frame}[allowdisplaybreaks]
	\frametitle{Criterios de buena retroalimentación \cite{susan20}}
	Susan Brookhart (2020) menciona los mismo elemento de una retroalimentación efectiva que vimos en Hattie y Heritage, pero ella hace enfásis en que la calidad de esta está dada por:
	\begin{itemize}
		\small
		\justifying
		\item Específica y descriptiva: No debe ser sólo evaluativa (ej. “bien”, “excelente”), sino señalar qué hizo bien el estudiante o qué necesita mejorar.
		\item Relacionada con criterios explícitos: Debe estar anclada a los objetivos de aprendizaje, no ser vaga o solo basada en intuición.
		\item Procesable (accionable): El estudiante debe saber qué hacer a continuación. Si no puede usarla, no sirve.
		\item Oportuna: La retroalimentación debe darse lo suficientemente pronto como para influir en el desempeño.
		\item Equilibrada (constructiva y motivadora): Debe combinar reconocimiento del logro con orientación para mejorar.
		\item Usar lenguaje comprensible: Evitar jerga, tecnicismos o generalidades. El foco está en la claridad.
	\end{itemize}
\end{frame}

\begin{frame}[allowframebreaks]
	\frametitle{Evaluación ``para'' el aprendizaje \cite{william08}}

Para Wiliam, la retroalimentación es solo una parte de un ecosistema pedagógico más amplio llamado ``Evaluación para el Aprendizaje'', centrado en la activación del aprendizaje durante el proceso (no al final).

\begin{itemize}
\item Clarificar, compartir y entender los objetivos de aprendizaje y criterios de éxito.
\item Diseñar actividades que generen evidencia del aprendizaje.
\item Proporcionar \textbf{retroalimentación} que impulse el aprendizaje.
\item Activar a los estudiantes como recursos de aprendizaje entre ellos [\textbf{andamiaje}].
\item Activar a los estudiantes como agentes de su propio aprendizaje [\textbf{metacognición}].

\end{itemize}

Algunas observaciones de William y col.
\begin{itemize}

\item La retroalimentación no es efectiva por sí sola: su efectividad depende de cómo el estudiante la recibe y usa.

\item Una retroalimentación puede ser perfectamente formulada y, sin embargo, tener cero impacto si el estudiante no la comprende o no la usa.

\item Critica la “retroalimentación correctiva” que no involucra al estudiante.

\end{itemize}
\end{frame}

\begin{frame}{Ejercicio en clase}
	Usted se encuentra trabajando con sus estudiantes el modelo de \textit{explicación científica óntico} (mecanicista) junto con el contenido: \textit{estrategias evolutivas de supervivencia}. Como contexto tiene la siguiente observación:
	\begin{itemize}[<+->]
		\item Profesor: ``Explique por qué, en general, los roedores tienen muchas crías''
		\item Estudiante:  ``Porque se aparean mucho profesor''
	\end{itemize}
	\visible<3->{¿Qué le diría al estudiante como retroalimentación? Siga los criterios de calidad y la estructura de una retroalimentación. }

\end{frame}

\section{Bibliografía}

\begin{frame}[allowframebreaks]{Bibliografía}

	\bibliography{../bibliografia}

\end{frame}

{
\setbeamertemplate{navigation symbols}{}
\begin{frame}
	\makebox[\linewidth]{\includegraphics[width=\paperwidth]{../template/background-last-page}}
\end{frame}
}


\end{document}
