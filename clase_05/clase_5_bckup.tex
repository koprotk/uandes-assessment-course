\documentclass[11pt, aspectratio=169, xcolor=table]{beamer} %aspectratio=169 presentación en 16:9; xcolor=table usar color en las tablas
\usepackage[utf8]{inputenc}
\usepackage[T1]{fontenc}
\usepackage{lmodern}
\usepackage[spanish]{babel}
\usetheme{Cuerna}
\usecolortheme{bluesimplex} %color del paquete
\usepackage{graphicx} %incluír imágenes
\usepackage{hyperref} %agregar urls
\usepackage{ragged2e} %alinear a la izquierda.
% \setbeamertemplate{bibliography item}[text] Colocar números en lugar de íconos.
\bibliographystyle{../apalike-es} %archivo de estilo apa en español en el directorio local, otros estilos: plain.
\usepackage{tikz} % Para dibujar 
\usetikzlibrary{shapes.geometric, arrows} % para hacer dibujos tipo diagramas de flujo y manejar la posición
\usepackage{pgfplots}

\begin{document}
	\author{Prof. Daniel Muñoz \\ 
		\href{mailto:dmunoz@miuandes.cl}{\texttt{dmunoz@miuandes.cl}}}
	\title{Taxonomías en evaluación educativa}
	%\subtitle{subtitulo}
	\logo{
		\includegraphics[width=1.8cm]{../marca_principal.png}
	}
	\institute{Facultad de Educación, Universidad de los Andes}
	\date{8 de abril de 2022}%today{}} %agregar el día de hoy
	%\subject{subject}
	%\setbeamercovered{transparent}
	%\setbeamertemplate{navigation symbols}{}
	\begin{frame}[plain]
		\setbeamertemplate{logo}{} %eliminar logo de la portada
		\maketitle
	\end{frame}

	\section{Resultados del instrumento}
	
	\begin{frame}{Análisis de resultados: Edumétrico}
		\begin{columns}
			\begin{column}{.5\linewidth}
				\begin{block}{Objetivos Módulo 1}
					\begin{enumerate}
						\justifying
						\item Comprender la teoría e historia de la evaluación de aprendizajes
						\item Entender que la evaluación es un proceso técnico, pero por sobre todo una actividad ética.
						\item Aplicar la teoría evaluativa en contextos a partir de testimonios reales o simulados.
					\end{enumerate}
				\end{block}
			\end{column}
			\begin{column}{.5\linewidth}
				\begin{tikzpicture}
					\begin{axis}[width=\linewidth,
						symbolic x coords={OA1, OA2, OA3},
						xtick=data, ylabel=Porcentaje de logro, bar width=20, ymin=0, ymax=100
						]
						\addplot[ybar,fill=blue] coordinates {
							(OA1,   80)
							(OA2,   72)
							(OA3,   75)
						};
						\addplot[red,line legend,sharp plot,nodes near coords={},
						update limits=false,shorten >=-3mm,shorten <=-3mm] 
						coordinates {(OA1,70) (OA3,70)} 
						node[midway,below]{exigencia 70\%};
						\addplot[green,line legend,sharp plot,nodes near coords={},
						update limits=false,shorten >=-3mm,shorten <=-3mm] 
						coordinates {(OA1,60) (OA3,60)} 
						node[midway,below]{exigencia 60\%};
					\end{axis}
				\end{tikzpicture}
			\end{column}
		\end{columns}
	\end{frame}
	
	\section{Introducción}
	\begin{frame}{Marco conceptual: ¿Qué es una taxonomía? \cite{STATEU2021}}
		\begin{columns}
			\begin{column}{.5\textwidth}
				\begin{itemize}[<+->]
					\item Una \textit{taxonomía} es una herramienta para construir objetivos de aprendizaje.
					\item Nos permite tener un lenguaje común y concreto del desempeño a observar.
					\item Es una herramienta que da coherencia a la evaluación y la enseñanza.
					\item La primera de ellas fue la elaborado por Benjamin Bloom en 1950, en la Universidad de Chicago.
					\item Todo esto es motivado por las ideas de \textit{Ralph Tylor}.
				\end{itemize}
			\end{column}
			\begin{column}{.5\textwidth}
				\begin{figure}
					\includegraphics<1>[width=.8\textwidth]{../imagenes/objetivo.png}
					\includegraphics<2>[width=.8\textwidth]{../imagenes/same_language.jpg}
					\only<3>{
					\begin{tikzpicture}[node distance=2cm]
						\node (eva) [rectangle, rounded corners, minimum width=3cm, minimum height=1cm, text centered, draw=black] {Evaluación};
						\node (ens) [rectangle, rounded corners, minimum width=3cm, minimum height=1cm,text centered, draw=black, below of=eva] {Enseñanza};
						\draw [thick,<->,>=stealth] (eva) -- node[anchor=west] {Taxonomía} (ens);
					\end{tikzpicture}}		
					\includegraphics<4>[width=.4\textwidth]{../imagenes/bloom}
					\includegraphics<5>[width=\textwidth]{../imagenes/bloom_tyler}						
					\caption{
						\only<1>{Clásico test para medir tu IQ en los años 30 con ello \textit{evaluaban tu inteligencia}.}
						\only<2>{A veces creemos que basta con hablar el mismo idioma}
						\only<3>{Una taxonomía media entre la enseñanza y la evaluación}
						\only<4>{Benjamin Bloom (1913 - 1999) Creador de su famosa taxonomía junto a otro autores de EEUU}
						\only<5>{Cita encontrada en el \textit{Handbook Cognitive Domain} \cite[pág 4]{BLOOM1956}}
					}
				\end{figure}
			\end{column}
		\end{columns}
	\end{frame}

\section{Taxonomía revisada de Anderson y Kratwohl}

	\begin{frame}{Origenes de la Taxonomía de Bloom revisada\cite{KRATWOHL2002} (TBRAK)}
		\begin{columns}
			\begin{column}{.5\linewidth}
				\begin{itemize}[<+->]
					\item La taxonomía original de Bloom es una taxonomía del \textit{Dominio Cognitivo}.
					\item El \textit{Conocimiento} entendido por Bloom contenía ambos \textit{\color{red} lo que hacen} y \textit{\color{blue} lo que saben}. Anderson y Kratwohl separan ambos.
					\item La taxonomía revisada de Bloom es una taxonomía del \textit{Conocimiento} y \textit{Procesos Cognitivos}.
				\end{itemize}
			\end{column}
			\begin{column}{.5\linewidth}
				\only<1>{
					\begin{block}{Original}
						\begin{enumerate}
							\item Conocer
							\item Comprender
							\item Aplicar
							\item Analizar
							\item Sintetizar
							\item Evaluar
						\end{enumerate}
					\end{block}
				}
				\only<3>{
					\begin{tikzpicture}[node distance= 2.2cm]
						\node[rectangle, fill=gray!50] (org) {Original};
						\node[rectangle, rounded corners, minimum width=3cm, minimum height=1cm, text centered, draw=black, below of = org] (knowB) {Conocer};
						\node[rectangle, rounded corners, minimum width=3cm, minimum height=1cm, text centered, draw=black, below right of= knowB] (knowK) {Conocer};
						\node[rectangle, rounded corners, minimum width=3cm, minimum height=1cm, text centered, draw=black, below left of= knowB] (rem) {Recordar};
						\node[rectangle, fill=gray!50, below right of = rem] (rev ){Revisada};
						\draw[->,thick,>=stealth] (knowB)  -- node[anchor=west] {Conocimiento} (knowK);
						\draw[->,thick,>=stealth] (knowB) -- node[anchor=east] {Proceso Cognitivo} (rem);
					\end{tikzpicture}
				}
				\only<2>{
					``\color{red}Los estudiantes deben ser capaces de recordar \color{blue} la ley de oferta y demanda en economía'' 
				}
			\end{column}
		\end{columns}
	\end{frame}
	
	\begin{frame}{El conocimiento en la TBRAK}
		\begin{columns}
			\begin{column}{.5\linewidth}
				\only<1-7>{
				\begin{block}{Factual}
					Son los elementos básicos que un estudiante debe conocer para estar familiarizado con una disciplina o un problema a resolver en ella.
					\begin{itemize}[<+->]
						\item<2-> Conocimiento de la terminología.
						\item<3-> Conocimiento de detalles específicos y elementales.
					\end{itemize}
				\end{block}}
				\only<8->{
				\begin{block}{Procedimiental}
					Como hacer algo; métodos de investigación, y criterios para usar habilidades, algoritmos, técnicas y métodos.
					\begin{itemize}
						\item<9-> Conocimiento de habilidades, algoritmos, técnicas y métodos específicos de una materia.
						\item<10-> Conocimiento de los criterios de uso de un determinado procedimiento.
					\end{itemize}	
				\end{block}}
			\end{column}
			\begin{column}{.5\linewidth}
			\only<4-10>{
				\begin{block}{Conceptual}
					La interrelación entre los elementos en una gran estructura que se les da una función conjunta.
					\begin{itemize}
						\item<5-> Conocimiento de clasificaciones y categorías.
						\item<6-> Conocimiento de principios y generalizaciones.
						\item<7-> Conocimiento de teorías, modelos y estructuras.
					\end{itemize}
				\end{block}}
			\only<11->{
			\begin{block}{Metacognición \cite{CAMBRIDGE2021}}
				El conocimiento \textit{metacognitivo} en general es percatarse y conocer la propia cognición.
				\begin{itemize}
					\item<12-> Conocimiento de como abordar una tarea de aprendizaje.
					\item<13-> Auto-conocimiento 
				\end{itemize}
			\end{block}
			}
			\end{column}
		\end{columns}
	\end{frame}

	\begin{frame}{El proceso cognitivo en la TBRAK}
		\begin{columns}
			\begin{column}{.5\linewidth}
				\only<1-14>{
				\begin{block}<1->{Reconocer}
					Recuperar conocimiento relevante desde la memoria de largo-plazo.
					\begin{itemize}
						\item<2-> Reconociendo.
						\item<3-> Recordando.
					\end{itemize}
				\end{block}			
				\begin{block}<12->{Aplicar}
					Usar un procedimiento en determinada situación.
					\begin{itemize}
						\item<13-> Ejecutando.
						\item<14-> Implementando.
					\end{itemize}
				\end{block}}
				\only<15-21>{
				\begin{block}<15->{Analizar}
					Separar en sus componentes y determinar como estos se tributan en una sola unidad de sentido.
					\begin{itemize}
						\item<16-> Diferenciando
						\item<17-> Organizando
						\item<18-> Atribuyendo
					\end{itemize}
				\end{block}}
				\only<22->{
				\begin{block}{Crear}
					Agregar elementos en un todo innovador y coherente o desarrollando un nuevo producto.
					\begin{itemize}
						\item<23-> Generando
						\item<24-> Planeando
						\item<25-> Produciendo
					\end{itemize}
				\end{block}}
			\end{column}
			\begin{column}{.5\linewidth}
				\only<1-14>{
				\begin{block}<4->{Comprender}
					Determinar el significado de instrucciones, orales, escritas, y gráfica.
					\begin{itemize}
						\item<5-> Interpretando
						\item<6-> Ejemplificando
						\item<7-> Clasificando
						\item<8-> Resumiendo
						\item<9-> Infiriendo
						\item<10-> Comparando
						\item<11-> Explicando
					\end{itemize}
				\end{block}}
				\only<15->{
					\begin{block}<19->{Evaluar}
						Hacer juicios a partir de un criterio o un estándar.
						\begin{itemize}
							\item<20-> Chequeando.
							\item<21-> Criticando.
						\end{itemize}
					\end{block}}
			\end{column}
		\end{columns}
	\end{frame}

	\begin{frame}{Con esto analizaremos nuestros OA para la evaluación y la enseñanza}
		\justifying
		Finalmente con este conocimiento analizaremos nuestros OA a partir de \textit{sustantivos de contenido} y \textit{verbos} para determinar donde se ubica nuestro OA, es importante que entienda que los OA se pueden colocar en varias celdas de la tabla.
		\begin{center}
			\textbf{Proceso Cognitivo}
			\rowcolors[]{1}{blue!20}{blue!10}
			\begin{tabular}{r|c|c|c|c|c|c}
				\hline
				 \textbf{Conocimiento} & Recordar & Comprender & Aplicar & Analizar & Evaluar & Crear \\
				\hline
				 Factual &  &  &  &  &  &  \\
				\hline
				Conceptual &  &  &  &  &  &  \\
				\hline
				 Procedimental &  &  &  &  &  &  \\
				\hline
				Metacognitivo &  &  &  &  &  &  \\
				\hline
			\end{tabular}
		\end{center}
	\end{frame}

	\begin{frame}{Ejemplo}
		Ubique el siguiente OA13 en la \textit{tabla taxonómica}:
		\begin{exampleblock}{}
			\justifying
		\only<1>{	
			``Desarrollar modelos que expliquen la estereoquímica e isomería de compuestos orgánicos como la glucosa, identificando sus propiedades y su utilidad para los seres vivos.''}
		\only<2->{
			``{\color{red}Desarrollar} modelos que {\color{red}expliquen} la {\color{blue}estereoquímica} e {\color{blue}isomería} de compuestos orgánicos como la glucosa, {\color{red}identificando} sus {\color{blue}propiedades} y su {\color{blue}utilidad} para los seres vivos.''}
			\begin{flushright}
				\cite{MINEDUC2021}
			\end{flushright}
		\end{exampleblock}
		\begin{center}
			\textbf{\color{red}Proceso Cognitivo}
			\rowcolors[]{1}{blue!20}{blue!10}
			\begin{tabular}{r|c|c|c|c|c|c}
				\hline
				\textbf{\color{blue}Conocimiento} & Recordar & Comprender & Aplicar & Analizar & Evaluar & Crear \\
				\hline
				Factual & \only<3->{OA13} &  & &  &  &  \\
				\hline
				Conceptual &  & \only<3->{OA13} &  &  &  &  \\
				\hline
				Procedimental &  &  & \only<3->{OA13}  &  &  &  \\
				\hline
				Metacognitivo &  &  &  &  &  &  \\
				\hline
			\end{tabular}
		\end{center}
	\end{frame}

	\begin{frame}{Otras Taxonomías: Habilidades Cognitivolingüisticas \cite{JORBA2000}}
		\begin{columns}
			\begin{column}{.5\linewidth}
				\begin{block}{Definición}
					\textit{``Jorba (2000), plantea que las habilidades cognitivo-lingüísticas son aquellas que se activan para producir diferentes tipologías textuales, que además son transversales, pero que a su vez se concretan de manera diferenciada en cada una de las áreas curriculares''} \cite{CASTILLO2013}
				\end{block}
			\end{column}
			\begin{column}{.5\linewidth}
				\only<1-5>{
					\begin{block}{Habilidades Cognitivolingüistica}
						\begin{itemize}[<+->]
							\item Narrar
							\item Describir
							\item Definir
							\item Explicar
							\item Demostrar
						\end{itemize}
					\end{block}}
				\only<6->{
				\begin{figure}
					\includegraphics[width=\linewidth]{../imagenes/TBAK_THC}
					\caption{Integración entre TBAK y las Habilidades Cognitivolingüisticas}
				\end{figure}}
			\end{column}
		\end{columns}
	\end{frame}
	
	\begin{frame}{Evaluación Formativa}
		En Canvas se dejó una evaluación formativa que consisten en clasificar diferentes OA en la \textit{tabla taxonómica}. Resuélvala junto a sus compañeros de grupo para discutirla en el grupo curso.
	\end{frame}
	
	\section{Bibliografía}
	
	\begin{frame}[allowframebreaks]{Bibliografía}
		
		\bibliography{../bibliografia}
		
	\end{frame}
	
	
	
\end{document}