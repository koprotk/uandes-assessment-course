\documentclass[11pt, aspectratio=169, xcolor=table,hyphens]{beamer}
\usepackage[utf8]{inputenc}
\usepackage[T1]{fontenc}
\usepackage{lmodern}
\usepackage[spanish]{babel}
\usetheme{default}
\usepackage{pdfrender}
\usepackage{tikz}
\usepackage{graphicx}
\usepackage{xcolor}
\usepackage{hyperref}
\usepackage{cite} %permite salto de línea en referencias largas
\usepackage{ragged2e}
\usepackage{smartdiagram}
\bibliographystyle{../apalike-es} %archivo de estilo apa ../apa-es en español en el directorio
\usepackage{pgfplots}
\usepackage{pgf-pie}
\usepackage[]{wasysym}
\usepackage{tabularx} % para ajuste del tamaño de la tabla
\usepackage{multirow} % juntar filas en una tabla
\usepackage{chemfig}
\usepackage{chemmacros}

\newcommand\encircle[1]{% Poder crear letras dentro de círculos
	\tikz[baseline=(X.base)]
	\node (X) [draw, shape=circle, inner sep=0] {\strut #1};}

\usepackage{../template/uandes169template}

%%%%%%%%%%%%%%%%%%%%%%%%%%%%%% INICIO DEL CONTENIDO %%%%%%%%%%%%%%%%%%%%%%%%%%%%%%%%%%%%%%%%%%%%%%

\begin{document}
\author{Prof. Daniel Muñoz \\
	\href{mailto:dmunoz@miuandes.cl}{\texttt{dmunoz@miuandes.cl}}}
\title{Puntuación y construcción de instrumentos}
\subtitle{Clase 11}
%\subject{subject}
%\setbeamercovered{transparent}
%\setbeamertemplate{navigation symbols}{}
\maketitle

\section{Resumen}

\subsection{Técnicas de Observación e Interrogación}
\begin{frame}{Técnicas evaluativas: Observación}
	\begin{columns}
		\begin{column}{.5\linewidth}
			\begin{block}{Técnicas de Observación}
				El sujeto posee una actitud pasiva frente al evaluador y el instrumento. Este tipo de técnicas se usan con frecuencia para registrar el \textit{día a día}.
				\begin{itemize}
					\item sistematizada
					      \begin{itemize}
						      \item Listas de cotejo
						      \item<2-> escalas de estimación
					      \end{itemize}
					\item no sistematizada
					      \begin{itemize}
						      \item<3-> notas de campo.
						      \item<4-> registros anecdóticos.
						      \item<5-> diario.
						      \item<6-> Registro de evidencias.
					      \end{itemize}
				\end{itemize}
			\end{block}
		\end{column}
		\begin{column}{.5\linewidth}
			\begin{block}{Técnicas de Interrogación}
				\justifying
				Los instrumentos de interrogación dependen de una participación activa del sujeto en el proceso. Esta técnica nace de la \textit{mayéutica} socrática.
				\begin{itemize}
					\item<7-> Entrevistas: estructuradas, semi-estructuradas.
					\item<8-> Encuestas.
					\item<9-> Exámenes.
					\item<10-> Pruebas Objetivas.
					\item<11-> Test.
				\end{itemize}
			\end{block}
		\end{column}
	\end{columns}
\end{frame}

\subsection{Preguntas Cerradas}
\begin{frame}{Pregunta cerrada \cite{MIDE2021}}
	\begin{columns}
		\begin{column}{.5\linewidth}
			Los ítems de preguntas cerradas o repuesta única pueden ser:
			\begin{itemize}
				\item<1-> términos pareados
				\item<2-> completación.
				\item<3-> ordenamiento.
				\item<4-> verdadero y falso.
			\end{itemize}
		\end{column}
		\begin{column}{.5\linewidth}
			\only<1>{
				\begin{exampleblock}{Empareje los conceptos}
					\begin{columns}
						\begin{column}{.3\linewidth}
							Columna 1
							\begin{enumerate}
								\item Sodio
								\item Cloro
								\item Xenon
							\end{enumerate}
						\end{column}
						\begin{column}{.4\linewidth}
							Columna 2
							\begin{itemize}
								\item[\dots] es alcalino.
								\item[\dots] es boroídeo.
								\item[\dots] es halógeno.
								\item[\dots] es gas noble.
							\end{itemize}
						\end{column}
					\end{columns}
				\end{exampleblock}
				\begin{alertblock}{Recomendaciones}
					\begin{itemize}
						\small
						\justifying
						\item Usar conceptos relacionados y ordenados 	lógicamente en ambas columnas.
						\item Incluir más elementos en la segunda columna.
					\end{itemize}
				\end{alertblock}
			}
			\only<2>{
				\begin{exampleblock}{Complete la oración}
					\justifying
					Un ..... presenta carga nula, en cambio un ..... presenta carga positivo y por el contrario un ..... presenta carga negativa.
				\end{exampleblock}
				\begin{alertblock}{Recomendaciones}
					\begin{itemize}
						\small
						\justifying
						\item Los espacios deben tener la misma extensión.
						\item Usar un espacio por palabra a completar.
						\item Evitar frases extensas y/o ambiguas.
						\item Utilizar solamente cuando las posibilidades de respuestas son \textbf{muy limitadas}.
					\end{itemize}
				\end{alertblock}
			}
			\only<3>{
				\begin{exampleblock}{\justifying Ordene las siguiente particulas subatómicas según masa, siendo 1 la más liviana 3 la de mayor masa.}
					\begin{itemize}
						\item[\dots] quark.
						\item[\dots] protón.
						\item[\dots] electrón.
					\end{itemize}
				\end{exampleblock}
				\begin{alertblock}{Recomendaciones}
					\begin{itemize}
						\item El orden deben ser \textbf{único}.
						\item El orden debe responder a un criterio \textit{objetivo}.
					\end{itemize}
				\end{alertblock}
			}
			\only<4>{
				\begin{exampleblock}{\justifying Marque la V de verdadero o F de falso según corresponda a cada enunciado}
					\begin{tabularx}{\linewidth}{Xc}
						Cristobal Colón descubrió América en 1592.                        & \encircle{V} \encircle{F} \\
						Las carabelas de Colón eran: La Pinta, La Niña y La Santa Iglesia & \encircle{V} \encircle{F} \\
					\end{tabularx}
				\end{exampleblock}
			}
			\only<5>{
				\begin{alertblock}{Recomendaciones}
					\begin{itemize}
						\justifying
						\item Los enunciados no deben expresar opiniones y deben ser lo \textit{más objetivos} posibles.
						\item Solo una idea central por enunciado.
						\item No usar conceptos que impliquen ambigüedad (ej. frecuentemente, a veces)
						\item Evitar expresiones absolutas (jamás, siempre, nunca, todo el tiempo, etc.)
						\item Máxima claridad en los enunciados.
						\item Ordenar los enunciados de forma aleatoria.
					\end{itemize}
				\end{alertblock}
			}
		\end{column}
	\end{columns}
\end{frame}

\begin{frame}{Ítem preguntas cerradas: Opción múltiple \cite{MIDE2021}}
	\begin{columns}
		\begin{column}{.5\linewidth}
			\textbf<2>{Lee el siguiente enunciado:
				\begin{quote}
					``El átomo es un esfera de carga  neutra que presenta electrones incrustados en una gran base positiva``
				\end{quote}}
			\textbf<3>{¿Qué modelo atómico se describe en el enunciado anterior?}
			\textbf<4-5>{
				\begin{enumerate}
					\item[a)] \textbf<7>{Planetario.}
					\item[b)] \textbf<7>{Sommerfield.}
					\item[c)] \textbf<6>{Pan de pascua.}
					\item[d)] \textbf<7>{Mecano-cuántico.}
				\end{enumerate}}
		\end{column}
		\begin{column}{.5\linewidth}
			\only<2>{
				\begin{block}{Contexto}
					\justifying Sirve como punto de referencia para el enunciado, es optativo, puede ser: gráfico, escrito, audio o video.
				\end{block}
				\begin{alertblock}{Recomendaciones}
					\begin{itemize}
						\justifying
						\item Incluir contexto necesarios y verosímiles.
						\item El contexto no puede ser un elemento que de pistas de la respuesta. Por ello es necesario para responder.
					\end{itemize}
				\end{alertblock}
			}
			\only<3>{
				\begin{block}{Enunciado}
					\justifying Es la pregunta o tarea concreta que se le solicita al evaluado.
				\end{block}
				\begin{alertblock}{Recomendaciones}
					\begin{itemize}
						\justifying
						\small
						\item Claro y sin ambigüedades.
						\item Evitar afirmaciones negativo, pero en caso de usarlas destacar la negación.
						\item Idealmente el enunciado (más contexto) debe permitir generar la respuesta del estudiante sin ver las alternativas.
						\item No usar \textit{doble proceso}. a) Solo I, b) Solo II, c) I y II, d) II y III, etc.
					\end{itemize}
				\end{alertblock}
			}
			\only<4>{
				\begin{block}{Alternativas}
					\justifying Es el número de respuestas \textbf{plausibles} para el enunciado (más contexto).
				\end{block}}
			\only<5>{
				\begin{alertblock}{Recomendaciones}
					\begin{itemize}
						\justifying
						\small
						\item Cuatro alternativas: una correcta y  tres incorrectas.
						\item Misma estructura gramatical mismos modos y tiempos verbales.
						\item Directas e independientes unas de otras.
						\item Idéntica lógica y/o contenido.
						\item Evitar el uso de \textit{ninguna de las anteriores} o similares.
						\item Número de caracteres similares.
						\item \textbf{Ordenarlas lógicamente: menor a mayor, menos extensa a más extensa. Error más común de ustedes}
					\end{itemize}
				\end{alertblock}}
			\only<6>{
				\begin{block}{Clave o alternativa correcta}
					La clave debe ser única y totalmente \textit{objetiva}.
				\end{block}
				\begin{alertblock}{Recomendaciones}
					\begin{itemize}
						\item Debe responder al enunciado.
						\item No debe ser de una lógica diferente al resto.
						\item Extensión similar al resto.
						\item Evitar marcas textuales o conceptos próximos entre el enunciado y la clave.
					\end{itemize}
				\end{alertblock}
			}
			\only<7>{
				\begin{block}{Distractores}
					Aquellas alternativas que sirven para diferenciar a un estudiante del que no. Estas deben ser plausibles. \textit{objetiva}.
				\end{block}
				\begin{alertblock}{Recomendaciones}
					\begin{itemize}
						\justifying
						\item Basarse en ideas previas.
						\item Incluir errores conceptuales.
						\item Usar creencias comunes o de dominio popular.
						\item Usar intervenciones sus estudiantes para construirlas.
					\end{itemize}
				\end{alertblock}
			}
		\end{column}
	\end{columns}
\end{frame}

\begin{frame}{Ojo con la construcción del ítem.}
	\begin{columns}
		\begin{column}{.5\linewidth}
			\begin{block}{Si tiene algo así}
				¿Qué tipo de juego usan en familia?
				\begin{itemize}
					\item[a)] juegos de mesa.
					\item[b)] juegos de video.
					\item[c)] juegos de cartas.
					\item[d)] juegos de tablero.
				\end{itemize}
			\end{block}
		\end{column}
		\begin{column}{.5\linewidth}
			\begin{block}{Déjelo así}
				¿Qué tipo de juego usan en familia? Juegos de:
				\begin{itemize}
					\item[a)] mesa.
					\item[b)] video.
					\item[c)] cartas.
					\item[d)] tablero.
				\end{itemize}
			\end{block}
		\end{column}
	\end{columns}
\end{frame}

\subsection{Preguntas  Abierta}
\begin{frame}{Creación de preguntas abiertas o de respuesta construida}
	\begin{columns}
		\begin{column}{.5\linewidth}
			\begin{block}{¿Cuándo usar preguntas abiertas?}
				\begin{itemize}
					\justifying
					\item Cuando debemos evaluar indicadores de alto nivel cognitivo (no excluyente).
					\item Cuando existen múltiples respuestas correctamente válidas para un desempeño.
					\item Considerar solamente si es capaz de codificar las respuestas posibles, para asegurar la objetividad en la corrección.
				\end{itemize}
			\end{block}
		\end{column}
		\begin{column}{.5\linewidth}
			\begin{block}{Ventajas \checkmark y Desventajas $\times$}
				\begin{itemize}
					\item[\checkmark] Se desempeñan bien en una gran cantidad de contextos.
					\item[\checkmark] Permiten medir desempeños complejos.
					\item[\checkmark] Altamente reutilizables.
					\item[$\times$] Lenta corrección.
					\item[$\times$] Difíciles de construir.
				\end{itemize}
			\end{block}
		\end{column}
	\end{columns}
\end{frame}

\begin{frame}{Preguntas abiertas: Ejemplos y Recomendaciones}
	\begin{columns}
		\begin{column}{.5\linewidth}
			\begin{exampleblock}{Ejemplos}
				\begin{itemize}
					\item Dibuje un mapa conceptual la clasificación de los diferentes tipos de evaluación educativa.
					\item Escriba una fábula que incluya 1 personaje principal y 2 secundarios en el espacio asignado.
					\item Justifique porque el ácido se vierte sobre el agua como norma de laboratorio y no al revés.
				\end{itemize}
			\end{exampleblock}
		\end{column}
		\begin{column}{.5\linewidth}
			\begin{block}{Recomendaciones}
				\begin{itemize}
					\item Usar un lenguaje claro, para ello:
					      \begin{itemize}
						      \item Usar verbos que impliquen acciones precisas (Dibuje, Calcule, compare, diseñe, etc.)
						      \item Explicitar el nivel de detalle, ejemplo: Entregue dos ejemplos.
					      \end{itemize}
					\item Evitar inclusión de elementos innecesarios en la pregunta. Usar contextos solamente cuando sea necesario.
				\end{itemize}
			\end{block}
		\end{column}
	\end{columns}
\end{frame}

\begin{frame}{Rubricas}
	\begin{columns}
		\begin{column}{.5\linewidth}
			\begin{itemize}[<+->]
				\item Las preguntas abiertas nos permiten recoger información que nos entrega el estudiante. Pero a diferencia de las preguntas cerradas, estas no nos indican como clasificar al estudiante con su respuesta.
				\item Entonces siempre que construimos una pregunta abierta también debemos construir su \textit{rúbricas} respectiva.
			\end{itemize}
		\end{column}
		\begin{column}{.5\linewidth}
			\begin{block}<3>{La rúbrica}
				\begin{itemize}
					\item Una rúbrica es un instrumento que nos permite clasificar los desempeños de los sujetos evaluados.
					\item En general toda rúbrica es una tabla que posee filas y columnas.
					\item Generalmente:
					      \begin{itemize}
						      \item Las columnas son los niveles de logro que obtiene el estudiante
						      \item Las filas son las dimensiones a evaluar.
					      \end{itemize}
				\end{itemize}
			\end{block}
		\end{column}
	\end{columns}
\end{frame}

\begin{frame}{Tipos de Rúbricas: Holística \cite{CRAIG2000}}
	\begin{columns}
		\begin{column}{.5\linewidth}
			\begin{block}{Holística}
				Rubricas donde el desempeño no necesita subdividirse y por tanto basta con describir los diferentes niveles de logro desde el nivel máximo de logro hasta el nivel ausente.
			\end{block}
			\begin{alertblock}<3->{$\rightarrow$}
				Este ejemplo es solo una guía de formato.
			\end{alertblock}
		\end{column}
		\begin{column}{.5\linewidth}
			\only<1>{
				\begin{tabularx}{\linewidth}{|c|X|}
					\hline
					Puntaje & Descripción                                                        \\
					\hline
					3       & Demuestra un entendimiento total del problema                      \\
					\hline
					2       & Demuestra un entendimiento parcial del problema                    \\
					\hline
					1       & no demuestra un entendimiento del problema                         \\
					\hline
					0       & no entrega la tarea o esta no tiene nada que ver con lo solicitado \\
					\hline
				\end{tabularx}}
			\only<2->{\footnotesize
				\begin{tabularx}{\linewidth}{|c|X|X|}
					\hline
					Puntaje & Descripción                                                        & Ejemplo                           \\
					\hline
					3       & Demuestra un entendimiento total del problema                      & Ejemplo de respuesta máxima       \\
					\hline
					2       & Demuestra un entendimiento parcial del problema                    & Ejemplo de respuesta insuficiente \\
					\hline
					1       & no demuestra un entendimiento del problema                         & Ejemplo de respuesta errónea.     \\
					\hline
					0       & no entrega la tarea o esta no tiene nada que ver con lo solicitado & Sin ejemplo                       \\
					\hline
				\end{tabularx}}
		\end{column}
	\end{columns}
\end{frame}

\begin{frame}{Rúbrica Analítica}
	\begin{block}{Análitica}
		Las rúbrica analíticas se usan cuando el desempeño puede subdividirse en dimensiones independientes. Personalmente es el tipo de rúbrica que recomiendo.
	\end{block}
	\begin{exampleblock}{Ejemplo}
		Por espacio en la presentación no puedo mostrar rúbricas analíticas pero aquí hay algunos ejemplos destacables:
		\begin{itemize}
			\item \url{http://rubistar.4teachers.org/index.php?screen=ShowRubric&rubric_id=2821442&}
			\item \url{https://www.ses.unam.mx/curso2011/pdf/RubricaEnsayo.pdf}
		\end{itemize}
	\end{exampleblock}

\end{frame}

\section{Instrumentos de evaluación}

\begin{frame}{Asignación de puntajes para las preguntas}
	\begin{columns}
		\begin{column}{.5\linewidth}
			Construir un instrumento de evaluación no es tarea sencilla.
			\begin{enumerate}[<+->]
				\item Como habrá visto todo inicia con la tabla de especificaciones (TE).
				\item Se construirán las preguntas siguiendo las indicaciones de la TE.
				\item Se califican las preguntas siguiendo diversos métodos dependiendo del tipo de ítem.
			\end{enumerate}
		\end{column}
		\begin{column}{.5\linewidth}
			\begin{figure}
				\only<1-2>{
					\begin{tabular}{|c|c|c|c|c|}
						\hline
						OA                   & IE    & Hab. & \# P & Dif.       \\
						\hline
						\multirow{3}{*}{OA1} & IE1.1 & H1.1 & 5    & 2F, 2M, 1D \\
						\cline{2-5}
						                     & IE1.2 & H1.2 & 3    & 2M, 1D     \\
						\cline{2-5}
						                     & IE1.3 & H1.3 & 2    & 2D         \\
						\hline
						\multirow{2}{*}{OA2} & IE2.1 & H2.1 & 4    & 2F, 2M     \\
						\cline{2-5}
						                     & IE2.2 & H2.2 & 2    & 1M         \\
						\cline{2-5}
						                     & IE2.3 & H2.3 & 1    & 1D         \\
						\hline
					\end{tabular}}
				\only<3>{
					\begin{tikzpicture}[node distance= 2.2cm]
						\node[rectangle, rounded corners, minimum width=3cm, minimum height=1cm, text centered, draw=black] (knowB) {Pregunta};
						\node[rectangle, rounded corners, minimum width=3cm, minimum height=1cm, text centered, draw=black, below right of= knowB] (knowK) {Abierta};
						\node[rectangle, rounded corners, minimum width=3cm, minimum height=1cm, text centered, draw=black, below left of= knowB] (rem) {Cerrada};
						\draw[->,thick,>=stealth] (knowB)  -- (knowK);
						\draw[->,thick,>=stealth] (knowB) -- (rem);
					\end{tikzpicture}}
				\caption{
					\only<1-2>{Tabla de especificaciones}
					\only<3>{Tipos de ítem}
				}
			\end{figure}
		\end{column}
	\end{columns}
\end{frame}

\begin{frame}{Calificación de preguntas}
	\begin{columns}
		\begin{column}{.5\linewidth}
			\begin{itemize}
				\item<1-> Si el ítem es cerrado cada ítem, frase, ordenamiento correcto tendrá siempre 1 punto:
				\item<6-> Si el ítem es abierto, la puntuación de la pregunta es igual a la puntuación de la rúbrica.
				\item<7-> Se recomienda que el salto entre niveles y dimensiones sea siempre de 1 punto, a menos que pueda justificar otra decisión.
			\end{itemize}
		\end{column}
		\begin{column}{.5\linewidth}
			\only<1>{
				\begin{block}{1 punto}
					¿Qué tipo de juego usan en familia? Juegos de:
					\begin{itemize}
						\item[a)] mesa.
						\item[b)] video.
						\item[c)] cartas.
						\item[d)] tablero.
					\end{itemize}
				\end{block}}
			\only<2>{
				\begin{block}{3 puntos}
					\justifying Ordene las siguiente particulas subatómicas según masa, siendo 1 la más liviana 3 la de mayor masa.
					\begin{itemize}
						\item[\dots] quark.
						\item[\dots] protón.
						\item[\dots] electrón.
					\end{itemize}
				\end{block}
			}
			\only<3>{
				\begin{block}{2 puntos}
					Marque la V de verdadero o F de falso según corresponda a cada enunciado
					\begin{tabularx}{\linewidth}{Xc}
						Cristobal Colón descubrió América en 1592.                        & \encircle{V} \encircle{F} \\
						Las carabelas de Colón eran: La Pinta, La Niña y La Santa Iglesia & \encircle{V} \encircle{F} \\
					\end{tabularx}
				\end{block}
			}
			\only<4>{
				\begin{block}{3 puntos}
					Complete la oración\\
					\justifying
					Un ..... presenta carga nula, en cambio un ..... presenta carga positivo y por el contrario un ..... presenta carga negativa.
				\end{block}
			}
			\only<5>{
				\begin{block}{3 puntos}
					Empareje los conceptos
					\begin{columns}
						\begin{column}{.3\linewidth}
							Columna 1
							\begin{enumerate}
								\item Sodio
								\item Cloro
								\item Xenon
							\end{enumerate}
						\end{column}
						\begin{column}{.4\linewidth}
							Columna 2
							\begin{itemize}
								\item[\dots] es alcalino.
								\item[\dots] es boroídeo.
								\item[\dots] es halógeno.
								\item[\dots] es gas noble.
							\end{itemize}
						\end{column}
					\end{columns}
				\end{block}
			}
			\only<6->{
				Puntos totales: 12 puntos.
				\begin{tabular}{|c|c|c|c|c|}
					\hline
					   & L max & L medio & L min & NL  \\
					\hline
					D1 & 3pt   & 2pt     & 1pt   & 0pt \\
					\hline
					D2 & 3pt   & 2pt     & 1pt   & 0pt \\
					\hline
					D3 & 3pt   & 2pt     & 1pt   & 0pt \\
					\hline
					D4 & 3pt   & 2pt     & 1pt   & 0pt \\
					\hline
				\end{tabular}
			}
		\end{column}
	\end{columns}
\end{frame}

\begin{frame}{Puntuación de la prueba}
	\begin{columns}
		\begin{column}{.5\linewidth}
			La puntuación total de la prueba dependerá de:
			\begin{itemize}[<+->]
				\item la planificación de la enseñanza.
				\item y lo enseñado
				\item<9->[OJO] Si naturalmente los puntajes del instrumento no se alinean con la enseñanza, deberá forzar mediante ponderación.
			\end{itemize}
		\end{column}
		\begin{column}{.5\linewidth}
			\begin{exampleblock}{Ejemplo}
				\begin{itemize}[<+->]
					\item Clases planificadas 10 para 2 OA
					      \begin{itemize}
						      \item 7 OA1
						      \item 3 OA2
					      \end{itemize}
					\item esto significa que el 70\% del puntaje de la prueba deben ser de ítems del OA1.
					\item y el otro 30\% del puntaje debe ser de ítems del OA2.
					\item[\checkmark] Siempre deberá ajustar sus instrumentos después de la enseñanza y antes aplicarlo.
				\end{itemize}
			\end{exampleblock}
		\end{column}
	\end{columns}
\end{frame}

\begin{frame}{Penalización de ítems de verdadero y falso y opción múltiple}
	\begin{columns}
		\begin{column}{.5\linewidth}
			\begin{itemize}
				\item Los ítems cerrados se penalizarán según la siguiente fórmula.
				\item Si es verdadero y falso, la penalización tiene la siguiente fórmula.
				\item Si es de selección múltiple será
			\end{itemize}

		\end{column}
		\begin{column}{.5\linewidth}
			\begin{equation}
				puntaje = aciertos - \frac{errores}{n-1}
			\end{equation}
			\begin{equation}
				puntaje = aciertos - \frac{errores}{2-1}
			\end{equation}
			\begin{equation}
				puntaje = aciertos - \frac{errores}{4-1}
			\end{equation}
		\end{column}
	\end{columns}
\end{frame}


\begin{frame}{Ensamblaje del instrumento |}
	Una vez, tenga los ítems desarrollados, deberá ensamblar el instrumento, para ello:
	\begin{enumerate}[<+->]
		\item Agregar un título al instrumento, una zona de identificación del o los sujeto (si amerita), el puntaje total del instrumento.
		\item Agruparlos por tipo cerrados y abiertos.
		\item Dentro de los cerrados agrupar según clase: términos pareados, completación, ordenamiento y verdadero y falso.
		\item Ordenar las clases según dificultad promedio del grupo.
		\item Dentro de cada clase ordenarlos de menor dificultad a mayor dificultad.
		\item Los ítems cerrados siempre irán primero que los abiertos.
		\item Agregar instrucciones para cada grupo de ítems (abiertos y cerrados) y para cada grupo clase.
	\end{enumerate}
\end{frame}

\begin{frame}{Ensamblaje del instrumento ||}
	\begin{enumerate}[<+->]
		\item[8] En cada instrucción ser claro y preciso de como espera que conteste el sujeto.
		\item[9] Para el caso de ítems cerrados informar del puntaje del grupo y el número de ítems y si existirá o no penalización y cómo se ejecutara.
		\item[10.] Para el caso de ítems abiertos agregar al final de cada enunciado el número de puntos de la pregunta.
	\end{enumerate}
\end{frame}



\section{Bibliografía}

\begin{frame}[allowframebreaks]{Bibliografía}

	\bibliography{../bibliografia}

\end{frame}

{
\setbeamertemplate{navigation symbols}{}
\begin{frame}
	\makebox[\linewidth]{\includegraphics[width=\paperwidth]{../template/background_last_page}}
\end{frame}
}



\end{document}
