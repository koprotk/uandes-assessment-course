\documentclass[11pt, aspectratio=169, xcolor=table,hyphens]{beamer}
\usepackage[utf8]{inputenc}
\usepackage[T1]{fontenc}
\usepackage{lmodern}
\usepackage[spanish]{babel}
\usetheme{default}
\usepackage{pdfrender}
\usepackage{tikz}
\usepackage{graphicx}
\usepackage{xcolor}
\usepackage{hyperref}
\usepackage{cite} %permite salto de línea en referencias largas
\usepackage{ragged2e}
\usepackage{smartdiagram}
\bibliographystyle{../apalike-es} %archivo de estilo apa ../apa-es en español en el directorio
\usepackage{pgfplots}
\usepackage[]{wasysym}

\usepackage{../template/uandes169template}

\begin{document}
\author{Prof. Daniel Muñoz \\
	\href{mailto:dmunoz@miuandes.cl}{\texttt{dmunoz@miuandes.cl}}}
\title{Construcción de Objetivos e indicadores}
\subtitle{Clase 6}
\date{16 de abril de 2025} %agregar el día de hoy
%\subject{subject}
%\setbeamercovered{transparent}
%\setbeamertemplate{navigation symbols}{}

\maketitle

\section{Resultados del proceso}

\begin{frame}{Análisis de resultados: Edumétrico}
	\begin{columns}
		\begin{column}{.5\linewidth}
			\begin{block}{Objetivos Módulo 1}
				\begin{enumerate}
					\justifying
					\item Comprender la teoría e historia de la evaluación de aprendizajes
					\item Entender que la evaluación es un proceso técnico, pero por sobre todo una actividad ética.
					\item Aplicar la teoría evaluativa en contextos a partir de testimonios reales o simulados.
				\end{enumerate}
			\end{block}
		\end{column}
		\begin{column}{.5\linewidth}
			\only<1>{
				\begin{tikzpicture}
					\begin{axis}[width=\linewidth,
							symbolic x coords={EF1, EF2, ES1},
							xtick=data, ylabel=Participación, bar width=20, ymin=0, ymax=100
						]
						\addplot[ybar,fill=blue] coordinates {
								(EF1,   45)
								(EF2,   42)
								(ES1,   98)
							};
						\addplot[red, line legend, sharp plot, nodes near coords={}, update limits=false,shorten >=-3mm,shorten <=-3mm]
						coordinates {(EF1, 60) (ES1, 60)}
						node[midway,below]{participación 60\%};
						]
					\end{axis}
				\end{tikzpicture}}
			\only<2>{
				\begin{tikzpicture}
					\begin{axis}[width=\linewidth,
							symbolic x coords={OA1, OA2, OA3},
							xtick=data, ylabel=Porcentaje de logro, bar width=20, ymin=0, ymax=100
						]
						\addplot[ybar,fill=blue] coordinates {
								(OA1,   61)
								(OA2,   66)
								(OA3,   63)
							};
						\addplot[red,line legend,sharp plot,nodes near coords={},
							update limits=false,shorten >=-3mm,shorten <=-3mm]
						coordinates {(OA1,60) (OA3,60)}
						node[midway,below]{exigencia 60\%};
					\end{axis}
				\end{tikzpicture}}
		\end{column}
	\end{columns}
\end{frame}

\section{Resumen}

\begin{frame}{Resumen: Módulo 2}
	\begin{columns}
		\begin{column}{.5\linewidth}
			\only<1>{\begin{block}{Taxonomía \cite{STATEU2021}}
					Recurso pedagógica que orienta a los equipos tener un lenguaje único sin ambigüedades
				\end{block}}
			\only<2>{\begin{block}{Taxonomía de Bloom \cite{BLOOM1956} }
					Primera taxonomía educativa originada por Benjamin Bloom (1913 - 1999), era una clasificación de habilidades (verbos) agrupadas en Dominios Cognitivos, del más elemental Conocimiento hasta la más compleja Evaluar.
				\end{block}}
			\only<3>{\begin{block}{Taxonomía Revisada de Anderson \cite{KRATWOHL2002} }
					En esta taxonomía se hace una revisión de la taxonomía de Bloom, convirtiéndola de unidimensional a bidimensional.
				\end{block}}
		\end{column}
		\begin{column}{.5\linewidth}
			\only<1>{
				\begin{figure}
					\includegraphics[width=.8\linewidth]{../imagenes/same_language}
					\caption{Siempre es bueno tener una taxonomía que oriente lo que hacemos}
				\end{figure}}
			\only<2>{
				\begin{block}{Taxonomía de Bloom}
					\begin{enumerate}
						\item Conocer
						\item Comprender
						\item Aplicar
						\item Analizar
						\item Sintetizar
						\item Evaluar
					\end{enumerate}
				\end{block}}
			\only<3>{
				\begin{block}{Taxonomía de Anderson \& Kratwohl}
					\begin{tikzpicture}[node distance= 2.2cm]
						\node[rectangle, fill=gray!50] (org) {Original};
						\node[rectangle, rounded corners, minimum width=3cm, minimum height=1cm, text centered, draw=black, below of = org] (knowB) {Conocer};
						\node[rectangle, rounded corners, minimum width=3cm, minimum height=1cm, text centered, draw=black, below right of= knowB] (knowK) {Conocer};
						\node[rectangle, rounded corners, minimum width=3cm, minimum height=1cm, text centered, draw=black, below left of= knowB] (rem) {Recordar};
						\node[rectangle, fill=gray!50, below right of = rem] (rev ){Revisada};
						\draw[->,thick,>=stealth] (knowB)  -- node[anchor=west] {Conocimiento} (knowK);
						\draw[->,thick,>=stealth] (knowB) -- node[anchor=east] {Proceso Cognitivo} (rem);
					\end{tikzpicture}
				\end{block}}
		\end{column}
	\end{columns}
\end{frame}

\section{Origenes}

\begin{frame}{Orígenes de la idea de Objetivos de Aprendizaje}
	\begin{columns}
		\begin{column}{.5\linewidth}
			\only<1>{\begin{block}{Segunda generación}
					Los orígenes de la idea de plantear objetivos en la evaluación viene de nuestro Ralph Tylor.
				\end{block}}
			\only<2-5>{\begin{block}{Objetivos, objetivos}
					\begin{itemize}
						\justifying
						\item<2-> En un principio un \textit{objetivo} eran exactamente los mismos en un proyecto cualquiera que en educación.
						\item<3-> Es por ello que usted encontrará la literatura (menos actualizada) algunas estrategias que exigen la observabilidad del objetivo.
						\item<5-> Actualmente para diferenciar los objetivos en educación de cualquier otro, lo complementamos con la palabra \textit{aprendizaje}.
					\end{itemize}
				\end{block}}
		\end{column}
		\begin{column}{.5\linewidth}
			\only<1-2>{
				\begin{figure}
					\includegraphics<1>[width=.6\linewidth]{../imagenes/tyler}
					\includegraphics<2>[width=.7\linewidth]{../imagenes/two_same}
					\caption{
						\only<1>{Ralph Tyler padre de la evaluación educativa}
						\only<2>{``Resistirse al entrenamiento no es lo mismo que el entrenamiento de resistencia''}
					}
				\end{figure}}
			\only<3>{
				\begin{block}{SMART \cite{STC2010}}
					\begin{itemize}
						\item \textbf{S}pecific: Sin dobles lecturas.
						\item \textbf{M}easurable: Medible.
						\item \textbf{A}chieved: realista y alcanzable.
						\item \textbf{R}elevant: relevante.
						\item \textbf{T}ime Bound: En un plazo.
					\end{itemize}
				\end{block}}
			\only<4>{
				\begin{block}{Pautas de Mager [2002]\nocite{EDUTEKA2002}}
					\begin{itemize}
						\item Audiencia: ¿Quién?
						\item Conducta: ¿Qué?
						\item Condición: ¿Cómo?
						\item Rango: ¿Cuánto?
					\end{itemize}
				\end{block}
			}
			\only<5>{
				\begin{block}{Entonces...}
					Un objetivo de \textit{aprendizaje} dependerá de lo que entendemos como \textit{aprendizaje}. ¿Qué es el aprendizaje?
				\end{block}
			}
		\end{column}
	\end{columns}
\end{frame}

\section{¿qué es el aprendizaje?}

\begin{frame}{¿Qué es el aprendizaje? \cite{SMITH2021}}
	\begin{columns}
		\begin{column}{.5\linewidth}
			Dependerá de qué teoría del aprendizaje nos posicionemos:
			\begin{block}<1->{Conductismo \only<6->{[Observable]}}
				Cambio en el comportamiento.
			\end{block}
			\begin{block}<2->{Humanismo \only<6->{[No observable]}}
				Un acto personal para alcanzar un potencial.
			\end{block}
		\end{column}
		\begin{column}{.5\linewidth}
			\begin{block}<3->{Teoría socio-cognitiva \only<6->{[No observable]}}
				Interacción con los demás en un contexto social
			\end{block}
			\begin{block}<4->{Cognitivismo \only<6->{[No observable]}}
				Proceso mental interno.
			\end{block}
			\begin{block}<5->{Constructivismo \only<6->{[No observable]}}
				Crear significado a partir de la experiencia
			\end{block}
		\end{column}
	\end{columns}
\end{frame}

\section{Construcción de Objetivos de Aprendizaje}

\begin{frame}{Construcción de Objetivos de Aprendizaje}
	\begin{columns}
		\begin{column}{.5\linewidth}
			\begin{block}<1->{Objetivo de Aprendizaje}
				Es usado para especificar un resultado anticipado que será logrado por un estudiante.
			\end{block}
		\end{column}
		\begin{column}{.5\linewidth}
			\begin{block}<2->{Fórmula}
				OA = {\color{red}Verbo en infinitivo} + {\color{blue}conocimiento}
			\end{block}
			\begin{exampleblock}<3->{Ejemplos}
				\begin{itemize}
					\item[OA1]<3-> {\color{red}Conocer} las {\color{blue}leyes de la termodinámica.}
					\item[OA2]<4-> {\color{red}Crear} una {\color{blue}planificación de estudios}.
					\item[OA3]<5-> {\color{red}Evaluar} una {\color{blue}noticia de actualidad}.
					\item[OA4]<6-> {\color{red}Reconocer} los diferentes {\color{blue}tipos de evaluación según agente}.
				\end{itemize}
			\end{exampleblock}
		\end{column}
	\end{columns}
\end{frame}

\section{Indicadores de evaluación}

\begin{frame}{Indicadores de evaluación}
	\begin{columns}
		\begin{column}{.5\linewidth}
			\begin{itemize}[<+->]
				\justifying
				\item Pero si los aprendizajes no son observables: ¿Cómo podemos saber si se logra o no?
				\item R: No podemos. Por tanto lo único que podemos hacer es \textit{inferir} su logro.
				\item Para ello supondremos que se cumplen sí sé logran ciertos criterios.
				\item Estos \textit{criterios de logro} deben ser observables.
			\end{itemize}
			\begin{block}<5->{Fórmula}
				IE = {\color{red}Verbo en modo indicativo} + {\color{cyan}conocimiento} + {\color{blue}condición}
			\end{block}
		\end{column}
		\begin{column}{.5\linewidth}
			\begin{block}<6->{OA4}
				{\color{red}Reconocer} los diferentes {\color{blue}tipos de evaluación según agente}.
				\begin{enumerate}
					\item<7-> {\color{red}Señalan} el {\color{cyan}tipo de evaluación según agente} {\color{blue}a partir relatos escritos}.
					\item<8-> {\color{red}Señalan} {\color{blue}un relato que describe} el {\color{cyan}tipo de evaluación según agente}.
					\item<9-> {\color{red}Definen} {\color{blue}por escrito} los {\color{cyan}tipos de evaluación según agente}.
					\item<10-> {\color{red}Listan} {\color{blue}por escrito} los {\color{cyan}tipos de evaluación según agente}.
				\end{enumerate}
			\end{block}
		\end{column}
	\end{columns}
\end{frame}

\begin{frame}{Síntesis \cite{WESTED2021}}
	\begin{columns}
		\begin{column}{.5\linewidth}
			\begin{itemize}[<+->]
				\item Los OA no son observables, los IE si.
				\item Los IE tributan a un OA. Por tanto muchos IE para un OA. y no al revés.
				\item En los OA utilizaremos verbos no observables: reconocer, comprender, aplicar, analizar, evaluar y crear.
				\item Por otro lado, los IE si deben ser observables, por tanto tendrán verbos como: explican, describen, dibujan, escriben, justifican, etc.
			\end{itemize}
		\end{column}
		\begin{column}{.5\linewidth}
			\begin{block}<5->{Considere que:}
				\begin{itemize}
					\justifying
					\item Ningún IE debe superar al OA desde los PC.
					\item Los OA se diferencian en bajo (Reconocer, comprender y aplicar) y alto nivel cognitivo (analizar, evaluar a crear)
					\item Los IE deben recorrer desde el PC:Reconocer al PC:OA. Siempre y cuando no existan otros OA que cubran esa necesidad.
					\item Cada OA debe contener a lo menos tres IE.
				\end{itemize}
			\end{block}
		\end{column}
	\end{columns}
\end{frame}

\begin{frame}{Instrumentos de orientación: Listas de cotejo}
	\begin{columns}
		\begin{column}{.5\linewidth}
			\begin{block}{El OA posee:}
				\begin{itemize}[<+->]
					\item[\Square] Un verbo no observable.
					\item[\Square] Un verbo en infinitivo.
					\item[\Square] Un conocimiento.
					\item[\Square] A lo menos tres IE.
				\end{itemize}
			\end{block}
		\end{column}
		\begin{column}{.5\linewidth}
			\begin{block}<5->{El IE posee:}
				\begin{itemize}
					\item[\Square]<6-> Un verbo en modo indicativo.
					\item[\Square]<7-> Un verbo observable.
					\item[\Square]<8-> Una condición.
					\item[\Square]<9-> Un conocimiento igual o relacionado al OA.
					\item[\Square]<10-> Un verbo inferior o igual al OA.
				\end{itemize}
			\end{block}
		\end{column}
	\end{columns}
\end{frame}

\begin{frame}{Tarea formativa}
	\justifying
	Como grupo busque seis objetivos de aprendizaje que desee usar para crear preguntas a partir de él. Estos objetivos pueden ser de: Syllabus de cursos que tengo o haya tenido, del currículum nacional o de cualquier otra fuente que disponga. Una vez los tenga seleccionados los discutiremos y trabajaremos la siguiente clase.
\end{frame}

%	\begin{frame}{Tarea formativa/sumativa}
%		\justifying
%		Ahora con lo aprendido, cree tres objetivos de aprendizaje (OA1, OA2, OA3) a partir de los Objetivos de Aprendizaje que seleccionó previamente cada uno con tres indicadores  de evaluación (1.1, 1.2, 1.3, 2.1, 2.2, 2.3, 3.1, 3.2, 3.3). Además los OA deben tributar a la siguiente tabla taxonómica:
%
%		\begin{center}
%		\rowcolors[]{1}{blue!20}{blue!10}
%		\begin{tabular}{|l|c|c|c|c|c|c|}
%			\hline
%			& Reconocer & Comprender & Aplicar & Analizar & Evaluar & Crear \\
%			\hline
%			Factual &  &  &  &  &  &  \\
%			\hline
%			Conceptual & OA1 &  &  & OA3 &  &  \\
%			\hline
%			Procedimental &  &  & OA2 &  &  &  \\
%			\hline
%			Metacognitivo &  &  &  &  &  &  \\
%			\hline
%		\end{tabular}
%		\end{center}
%	\end{frame}

\section{Bibliografía}

\begin{frame}[allowframebreaks]{Bibliografía}

	\bibliography{../bibliografia}

\end{frame}

{
\setbeamertemplate{navigation symbols}{}
\begin{frame}
	\makebox[\linewidth]{\includegraphics[width=\paperwidth]{../template/background-last-page}}
\end{frame}
}


\end{document}
