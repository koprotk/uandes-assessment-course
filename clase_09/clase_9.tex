\documentclass[11pt, aspectratio=169, xcolor=table,hyphens]{beamer}
\usepackage[utf8]{inputenc}
\usepackage[T1]{fontenc}
\usepackage{lmodern}
\usepackage[spanish]{babel}
\usepackage{pdfrender}
\usepackage{tikz}
\usepackage{graphicx}
\usepackage{xcolor}
\usepackage{hyperref}
\usepackage{cite} %permite salto de línea en referencias largas
\usepackage{ragged2e}
\usepackage{smartdiagram}
\bibliographystyle{../apalike-es} %archivo de estilo apa ../apa-es en español en el directorio
\usepackage{pgfplots}
\usepackage{pgf-pie}
\usepackage[]{wasysym}
\usepackage{tabularx} % para ajuste del tamaño de la tabla
\usepackage{multirow} % juntar filas en una tabla

\usepackage{../template/uandes169template}

\newcommand\encircle[1]{% Poder crear letras dentro de círculos
	\tikz[baseline=(X.base)]
	\node (X) [draw, shape=circle, inner sep=0] {\strut #1};}


%%%%%%%%%%%%%%%%%%%%%%%%%%%%%% INICIO DEL CONTENIDO %%%%%%%%%%%%%%%%%%%%%%%%%%%%%%%%%%%%%%%%%%%%%%

\begin{document}
\author{Prof. Daniel Muñoz \\
	\href{mailto:dmunoz@miuandes.cl}{\texttt{dmunoz@miuandes.cl}}}
\title{Técnicas de Observación, Interrogación y Preguntas Cerradas}
\subtitle{Clase 9}
\date{Miércoles 7 de mayo}
\maketitle

% \section{Encuesta de mitad de semestre}
% \begin{frame}{(2022) Encuesta de mitad de semestre: Parte Likert}
% 	\begin{columns}
% 		\begin{column}{.5\linewidth}
% 			\begin{itemize}[<+->]
% 				\item En esta asignatura estoy recibiendo retroalimentación oportuna que me ayuda a aprender.
% 				\item En esta asignatura recibo un trato personalizado que considera las circunstancias especiales que estamos viviendo.
% 				\item Existe una coherencia entre las evaluaciones y la metodología de trabajo.
% 				\item Mis profesores en esta asignatura se han adaptado bien a la modalidad online.
% 			\end{itemize}
% 		\end{column}
% 		\begin{column}{.5\linewidth}
% 			\begin{tikzpicture}
% 				\only<1>{
% 					\pie[radius=2]{
% 						88/Acuerdo,
% 						12/Desacuerdo}
% 				}
% 				\only<2>{
% 					\pie[radius=2]{
% 						81/Acuerdo,
% 						19/Desacuerdo
% 					}}
% 				\only<3>{
% 					\pie[radius=2]{
% 						88/Acuerdo,
% 						12/Desacuerdo
% 					}}
% 				\only<4>{
% 					\pie[radius=2]{
% 						100/Acuerdo,
% 						0/Desacuerdo
% 					}}
% 			\end{tikzpicture}
% 		\end{column}
% 	\end{columns}
% \end{frame}

% \begin{frame}{Encuesta de mitad de semestre: Mitad cualitativa}
% 	\begin{columns}
% 		\begin{column}{.5\linewidth}
% 			\begin{figure}
% 				\begin{tikzpicture}
% 					\pie[radius=1.8]{
% 						18/Retro.,
% 						9/Online,
% 						9/Presentaciones,
% 						9/Clase,
% 						55/Disposición
% 					}
% 				\end{tikzpicture}
% 				\caption{Lo Destacado, en general: La disposición a responder consultas}
% 			\end{figure}
% 		\end{column}
% 		\begin{column}{.5\linewidth}
% 			\begin{figure}
% 				\begin{tikzpicture}
% 					\pie[radius=1.8]{
% 						14/evaluación,
% 						14/validez semántica,
% 						14/retro. de clase,
% 						58/t. de exposición
% 					}
% 				\end{tikzpicture}
% 				\caption{Lo que debe mejorar, la duración de la clase expositiva}
% 			\end{figure}
% 		\end{column}
% 	\end{columns}
% \end{frame}

\section{Instrumentos de evaluación}
\begin{frame}{Instrumentos \cite{CASTILLO2010}}
	\begin{columns}
		\begin{column}{.5\linewidth}
			\begin{itemize}[<+->]
				\item Todo proceso evaluativo contempla la mediación de un instrumento.
				\item La selección del instrumento dependerá de múltiples criterios los cuales deben conjugarse en una \textbf{decisión}.
				\item Algunos ejemplos de instrumentos de evaluación para \textbf{diferentes técnicas}.
			\end{itemize}
		\end{column}
		\begin{column}{.5\linewidth}
			\only<1>{
				\begin{figure}
					\includegraphics[width=.7\linewidth]{../imagenes/assesment_process.jpg}
					\caption{Proceso evaluativo \cite{JMADISONU2021}}
				\end{figure}
			}
			\only<2>{
				\begin{itemize}
					\item Número de sujetos.
					\item Recursos económicos.
					\item Tiempo disponible.
					\item Objetivos de aprendizaje.
					\item Tipo de \textit{referente}.
					\item Modo (asincrónico/sincrónico).
					\item Tipo de \textit{agente}.
				\end{itemize}
			}
			\only<3>{
				\begin{columns}
					\begin{column}{.5\linewidth}
						\begin{itemize}
							\item Técnicas de Observación:
							      \begin{itemize}
								      \item Cuadernos.
								      \item Diarios.
								      \item Escalas de obs.
								      \item Listas de cotejo.
								      \item Registros anecdóticos.
							      \end{itemize}
						\end{itemize}
					\end{column}
					\begin{column}{.5\linewidth}
						\begin{itemize}
							\item Técnicas de interrogación:
							      \begin{itemize}
								      \item KPSI.
								      \item Exámenes.
								      \item Pruebas objetivas.
								      \item Pruebas estandarizadas.
								      \item Encuestas.
							      \end{itemize}
						\end{itemize}
					\end{column}
				\end{columns}
			}
		\end{column}
	\end{columns}
\end{frame}

\subsection{Técnicas de Observación}
\begin{frame}{Técnicas evaluativas: Observación}
	\begin{columns}
		\begin{column}{.5\linewidth}
			\justifying
			El sujeto posee una actitud pasiva frente al evaluador y el instrumento. Este tipo de técnicas se usan con frecuencia para registrar el \textit{día a día}.
			\begin{itemize}
				\item sistematizada
				      \begin{itemize}
					      \item Listas de cotejo
					      \item<2-> escalas de estimación
				      \end{itemize}
				\item no sistematizada
				      \begin{itemize}
					      \item<3-> notas de campo.
					      \item<4-> registros anecdóticos.
					      \item<5-> diario.
					      \item<6-> Registro de evidencias.
				      \end{itemize}
			\end{itemize}
		\end{column}
		\begin{column}{.5\linewidth}
			\begin{exampleblock}{Ejemplo}
				\only<1>{
					Nombre: Pedro Urdemales\\
					Curso: 4-A\\
					Asignatura: Lenguaje\\
					Fecha: 25-04-2021\\
					\begin{tabularx}{\linewidth}{|X|c|}
						\hline
						\textbf{Categoría}                                 & \textbf{Obs.} \\
						\hline
						Levanta la mano antes de intervenir                & \checkmark    \\
						\hline
						Se dirige al profesor con un lenguaje culto-formal & $\times$      \\
						\hline
						Mantiene un higiene personal adecuado              & \checkmark    \\
						\hline
					\end{tabularx}}
				\only<2>{
					Nombre: Pedro Urdemales\\
					Curso: 4-A\\
					Fecha: 25-04-2021\\
					Muy presente 4 3 2 1 0 Ausente
					\begin{tabularx}{\linewidth}{|X|c|}
						\hline
						\textbf{Categoría}                                 & \textbf{Obs.} \\
						\hline
						Levanta la mano antes de intervenir                & 4             \\
						\hline
						Se dirige al profesor con un lenguaje culto-formal & 0             \\
						\hline
						Mantiene un higiene personal adecuado              & 2             \\
						\hline
					\end{tabularx}
				}
				\only<3>{
					Fecha: 25-04-2021\\
					\justifying
					Se escribe respecto de lo que se observa, impresiones y reflexiones del docente respecto de lo que observa.
				}
				\only<4>{
					Fecha: 25-04-2021\\
					\justifying
					Se registra todo hecho totalmente anormal que observe el docente.
				}
				\only<5>{
					Fecha: 25-04-2021\\
					\justifying
					El docente registrará de forma diaria sus observaciones respecto de: el grupo curso, estudiantes y/o su propia práctica (diario docente).
				}
				\only<6>{
					Fecha: 25-04-2021\\
					Evidencia: Cuaderno del estudiante Pedro Urdemales\\
					\justifying
					Aquí se registrará lo observado respecto de alguna evidencia: cuadernos del estudiante, guías de trabajo, registros anecdóticos, diarios, etc.
				}
			\end{exampleblock}
		\end{column}
	\end{columns}
\end{frame}

\subsection{Técnicas de Interrogación}
\begin{frame}{Técnicas de Interrogación}
	\begin{columns}
		\begin{column}{.5\linewidth}
			\justifying
			Los instrumentos de interrogación dependen de una participación activa del sujeto en el proceso. Esta técnica nace de la \textit{mayéutica} socrática.
			\begin{itemize}
				\item<2-> Entrevistas: estructuradas, semi-estructuradas.
				\item<3-> Encuestas.
				\item<4-> Exámenes.
				\item<5-> Pruebas Objetivas.
				\item<6-> Test.
			\end{itemize}
		\end{column}
		\begin{column}{.5\linewidth}
			\only<1>{
				\begin{figure}
					\includegraphics[width=.8\linewidth]{../imagenes/mayeutica.jpg}
					\caption{\justifying proceso por el cual se devela el conocimiento del sujeto y enseña la verdad por medio de preguntas y respuesta}
				\end{figure}}
			\only<2>{
				\begin{block}{Entrevistas}
					\begin{itemize}\justifying
						\item \textbf{Instrumento:} Pautas de entrevistas.
						\item \textbf{ítems:} Preguntas abiertas.
						\item \textbf{¿Qué es?:} Conversación que sigue una pauta para develar información del sujeto. La diferencia entre la estructurada y semi-estructurada que la segunda admite por parte del entrevistado indagación adicional que no está en la \textit{pauta de entrevista}.
					\end{itemize}
				\end{block}
			}
			\only<3>{
				\begin{block}{Encuestas}
					\begin{itemize}
						\justifying
						\item \textbf{Instrumento:} Cuestionario (papel o digital).
						\item \textbf{ítems:} Escalas de apreciación, preguntas abiertas.
						\item \textbf{¿Qué es?:} Proceso de interrogación que busca saber el parecer de un sujeto respecto de un objeto para inferir por parte del evaluador alguna variable latente. valores, preferencias, actitudes, etc.
					\end{itemize}
				\end{block}
			}
			\only<4>{
				\begin{block}{Exámenes}
					\begin{itemize}
						\justifying
						\item \textbf{Instrumento:} examen escrito/online, examen oral/sincrónico.
						\item \textbf{ítems:} Preguntas cerradas y Preguntas abiertas.
						\item \textbf{¿Qué es?:} Proceso de interrogación que busca determinar el nivel de dominio de un desempeño de bajo y/o alto nivel cognitivo \textit{respecto de un criterio}.
					\end{itemize}
				\end{block}
			}
			\only<5>{
				\begin{block}{Pruebas Objetivas}
					\begin{itemize}
						\justifying
						\item \textbf{Instrumento:} examen escrito/online.
						\item \textbf{ítems:} Preguntas cerradas.
						\item \textbf{¿Qué es?:} Un caso especial de examen que es únicamente escrito y posee solamente preguntas cerradas.
					\end{itemize}
				\end{block}
			}
			\only<6>{
				\begin{block}{Tests}
					\begin{itemize}
						\justifying
						\item \textbf{Instrumento:} Prueba escrita/online.
						\item \textbf{ítems:} Preguntas cerradas.
						\item \textbf{¿Qué es?:} Proceso de interrogación que busca determinar la intensidad de la presencia de un atributo respecto de \textit{media de la muestra/población}.
					\end{itemize}
				\end{block}
			}
		\end{column}
	\end{columns}
\end{frame}

\subsubsection{Preguntas de Opción Múltiple}

\begin{frame}{Ítems: Pregunta cerrada \cite{MIDE2021}}
	\begin{columns}
		\begin{column}{.5\linewidth}
			Los ítems de preguntas cerradas o repuesta única pueden ser:
			\begin{itemize}
				\item<1-> términos pareados
				\item<2-> completación.
				\item<3-> ordenamiento.
				\item<4-> verdadero y falso.
			\end{itemize}
		\end{column}
		\begin{column}{.5\linewidth}
			\only<1>{
				\begin{exampleblock}{Empareje los conceptos}
					\begin{columns}
						\begin{column}{.3\linewidth}
							Columna 1
							\begin{enumerate}
								\item Sodio
								\item Cloro
								\item Xenón
							\end{enumerate}
						\end{column}
						\begin{column}{.4\linewidth}
							Columna 2
							\begin{itemize}
								\item[\dots] es alcalino.
								\item[\dots] es boroídeo.
								\item[\dots] es halógeno.
								\item[\dots] es gas noble.
							\end{itemize}
						\end{column}
					\end{columns}
				\end{exampleblock}
				\begin{alertblock}{Recomendaciones}
					\begin{itemize}
						\small
						\justifying
						\item Usar conceptos relacionados y ordenados 	lógicamente en ambas columnas.
						\item Incluir más elementos en la segunda columna.
					\end{itemize}
				\end{alertblock}
			}
			\only<2>{
				\begin{exampleblock}{Complete la oración}
					\justifying
					Un ..... presenta carga nula, en cambio un ..... presenta carga positivo y por el contrario un ..... presenta carga negativa.
				\end{exampleblock}
				\begin{alertblock}{Recomendaciones}
					\begin{itemize}
						\small
						\justifying
						\item Los espacios deben tener la misma extensión.
						\item Usar un espacio por palabra a completar.
						\item Evitar frases extensas y/o ambiguas.
						\item Utilizar solamente cuando las posibilidades de respuestas son \textbf{muy limitadas}.
					\end{itemize}
				\end{alertblock}
			}
			\only<3>{
				\begin{exampleblock}{\justifying Ordene las siguiente partículas subatómicas según masa, siendo 1 la más liviana 3 la de mayor masa.}
					\begin{itemize}
						\item[\dots] quark.
						\item[\dots] protón.
						\item[\dots] electrón.
					\end{itemize}
				\end{exampleblock}
				\begin{alertblock}{Recomendaciones}
					\begin{itemize}
						\item El orden deben ser \textbf{único}.
						\item El orden debe responder a un criterio \textit{objetivo}.
					\end{itemize}
				\end{alertblock}
			}
			\only<4>{
				\begin{exampleblock}{\justifying Marque la V de verdadero o F de falso según corresponda a cada enunciado}
					\begin{tabularx}{\linewidth}{Xc}
						Cristóbal Colón descubrió América en 1592.                        & \encircle{V} \encircle{F} \\
						Las carabelas de Colón eran: La Pinta, La Niña y La Santa Iglesia & \encircle{V} \encircle{F} \\
					\end{tabularx}
				\end{exampleblock}
			}
			\only<5>{
				\begin{alertblock}{Recomendaciones}
					\begin{itemize}
						\justifying
						\item Los enunciados no deben expresar opiniones y deben ser lo \textit{más objetivos} posibles.
						\item Solo una idea central por enunciado.
						\item No usar conceptos que impliquen ambigüedad (ej. frecuentemente, a veces)
						\item Evitar expresiones absolutas (jamás, siempre, nunca, todo el tiempo, etc.)
						\item Máxima claridad en los enunciados.
						\item Ordenar los enunciados de forma aleatoria.
					\end{itemize}
				\end{alertblock}
			}
		\end{column}
	\end{columns}
\end{frame}

\begin{frame}{Ítem preguntas cerradas: Opción múltiple \cite{MIDE2021}}
	\begin{columns}
		\begin{column}{.5\linewidth}
			\textbf<2>{Lee el siguiente enunciado:
				\begin{quote}
					``El átomo es un esfera de carga neutra que presenta electrones incrustados en una gran base positiva``
				\end{quote}}
			\textbf<3>{¿Qué modelo atómico se describe en el enunciado anterior?}
			\textbf<4-5>{
				\begin{enumerate}
					\item[a)] \textbf<7>{Planetario.}
					\item[b)] \textbf<7>{Sommerfield.}
					\item[c)] \textbf<6>{Pan de pascua.}
					\item[d)] \textbf<7>{Mecano-cuántico.}
				\end{enumerate}}
		\end{column}
		\begin{column}{.5\linewidth}
			\only<2>{
				\begin{block}{Contexto}
					\justifying Sirve como punto de referencia para el enunciado, es optativo, puede ser: gráfico, escrito, audio o vídeo.
				\end{block}
				\begin{alertblock}{Recomendaciones}
					\begin{itemize}
						\justifying
						\item Incluir contexto necesarios y verosímiles.
						\item El contexto no puede ser un elemento que de pistas de la respuesta. Por ello es necesario para responder.
					\end{itemize}
				\end{alertblock}
			}
			\only<3>{
				\begin{block}{Enunciado}
					\justifying Es la pregunta o tarea concreta que se le solicita al evaluado.
				\end{block}
				\begin{alertblock}{Recomendaciones}
					\begin{itemize}
						\justifying
						\small
						\item Claro y sin ambigüedades.
						\item Evitar afirmaciones negativo, pero en caso de usarlas destacar la negación.
						\item Idealmente el enunciado (más contexto) debe permitir generar la respuesta del estudiante sin ver las alternativas.
						\item No usar \textit{doble proceso}. a) Solo I, b) Solo II, c) I y II, d) II y III, etc.
					\end{itemize}
				\end{alertblock}
			}
			\only<4>{
				\begin{block}{Alternativas}
					\justifying Es el número de respuestas \textbf{plausibles} para el enunciado (más contexto).
				\end{block}}
			\only<5>{
				\begin{alertblock}{Recomendaciones}
					\begin{itemize}
						\justifying
						\small
						\item Cuatro alternativas: una correcta y  tres incorrectas.
						\item Misma estructura gramatical mismos modos y tiempos verbales.
						\item Directas e independientes unas de otras.
						\item Idéntica lógica y/o contenido.
						\item Evitar el uso de \textit{ninguna de las anteriores} o similares.
						\item Número de caracteres similares.
						\item Ordenarlas lógicamente: menor a mayor, menos extensa a más extensa.
					\end{itemize}
				\end{alertblock}}
			\only<6>{
				\begin{block}{Clave o alternativa correcta}
					La clave debe ser única y totalmente \textit{objetiva}.
				\end{block}
				\begin{alertblock}{Recomendaciones}
					\begin{itemize}
						\item Debe responder al enunciado.
						\item No debe ser de una lógica diferente al resto.
						\item Extensión similar al resto.
						\item Evitar marcas textuales o conceptos próximos entre el enunciado y la clave.
					\end{itemize}
				\end{alertblock}
			}
			\only<7>{
				\begin{block}{Distractores}
					Aquellas alternativas que sirven para diferenciar a un estudiante del que no. Estas deben ser plausibles. \textit{objetiva}.
				\end{block}
				\begin{alertblock}{Recomendaciones}
					\begin{itemize}
						\justifying
						\item Basarse en ideas previas.
						\item Incluir errores conceptuales.
						\item Usar creencias comunes o de dominio popular.
						\item Usar intervenciones sus estudiantes para construirlas.
					\end{itemize}
				\end{alertblock}
			}
		\end{column}
	\end{columns}
\end{frame}

\section{Evaluación Formativa}
\begin{frame}{Creación de ítems de opción múltiple}
	\justifying
	Ahora, individualmente, deberá \textbf{crear tres preguntas de opción múltiple para el indicador de evaluación IE1.1}, si no se puede ajustará el IE1.1 para que admita preguntas de selección múltiple de su tabla de especificaciones.
\end{frame}

\section{Bibliografía}

\begin{frame}[allowframebreaks]{Bibliografía}

	\bibliography{../bibliografia}

\end{frame}

{
\setbeamertemplate{navigation symbols}{}
\begin{frame}
	\makebox[\linewidth]{\includegraphics[width=\paperwidth]{../template/background-last-page}}
\end{frame}
}

\end{document}
