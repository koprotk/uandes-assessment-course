\documentclass[11pt, aspectratio=169, xcolor=table,hyphens]{beamer}
\usepackage[utf8]{inputenc}
\usepackage[T1]{fontenc}
\usepackage{lmodern}
\usepackage[spanish]{babel}
\usetheme{default}
\usepackage{pdfrender}
\usepackage{tikz}
\usepackage{graphicx}
\usepackage{xcolor}
\usepackage{hyperref}
\usepackage{cite} %permite salto de línea en referencias largas
\usepackage{ragged2e}
\usepackage{smartdiagram}
\bibliographystyle{../apalike-es} %archivo de estilo apa ../apa-es en español en el directorio
\usepackage{pgfplots}
\usepackage[]{wasysym}
\usepackage{tabularx} % para ajuste del tamaño de la tabla
\usepackage{multirow} % juntar filas en una tabla

\usepackage{../template/uandes169template}

\begin{document}
\author{Prof. Daniel Muñoz \\
	\href{mailto:dmunoz@miuandes.cl}{\texttt{dmunoz@miuandes.cl}}}
\title{Resumen del módulo}
\subtitle{Clase 8}
\date{24 de abril de 2024} %agregar el día de hoy
%\subject{subject}
%\setbeamercovered{transparent}
%\setbeamertemplate{navigation symbols}{}
\maketitle

\section{Resumen}

\begin{frame}{Resumen: Módulo 2}
	\begin{columns}
		\begin{column}{.5\linewidth}
			\only<1>{\begin{block}{Taxonomía \cite{STATEU2021}}
					Recurso pedagógica que orienta a los equipos tener un lenguaje único sin ambigüedades
				\end{block}}
			\only<2>{\begin{block}{Taxonomía de Bloom \cite{BLOOM1956} }
					Primera taxonomía educativa originada por Benjamin Bloom (1913 - 1999), era una clasificación de habilidades (verbos) agrupadas en Dominios Cognitivos, del más elemental Conocimiento hasta la más compleja Evaluar.
				\end{block}}
			\only<3>{\begin{block}{Taxonomía Revisada de Anderson \cite{KRATWOHL2002} }
					En esta taxonomía se hace una revisión de la taxonomía de Bloom, convirtiéndola de unidimensional a bidimensional.
				\end{block}}
		\end{column}
		\begin{column}{.5\linewidth}
			\only<1>{
				\begin{figure}
					\includegraphics[width=.8\linewidth]{../imagenes/same_language}
					\caption{Siempre es bueno tener una taxonomía que oriente lo que hacemos}
				\end{figure}}
			\only<2>{
				\begin{block}{Taxonomía de Bloom}
					\begin{enumerate}
						\item Conocer
						\item Comprender
						\item Aplicar
						\item Analizar
						\item Sintetizar
						\item Evaluar
					\end{enumerate}
				\end{block}}
			\only<3>{
				\begin{block}{Taxonomía de Anderson \& Kratwohl}
					\begin{tikzpicture}[node distance= 2.2cm]
						\node[rectangle, fill=gray!50] (org) {Original};
						\node[rectangle, rounded corners, minimum width=3cm, minimum height=1cm, text centered, draw=black, below of = org] (knowB) {Conocer};
						\node[rectangle, rounded corners, minimum width=3cm, minimum height=1cm, text centered, draw=black, below right of= knowB] (knowK) {Conocer};
						\node[rectangle, rounded corners, minimum width=3cm, minimum height=1cm, text centered, draw=black, below left of= knowB] (rem) {Recordar};
						\node[rectangle, fill=gray!50, below right of = rem] (rev ){Revisada};
						\draw[->,thick,>=stealth] (knowB)  -- node[anchor=west] {Conocimiento} (knowK);
						\draw[->,thick,>=stealth] (knowB) -- node[anchor=east] {Proceso Cognitivo} (rem);
					\end{tikzpicture}
				\end{block}}
		\end{column}
	\end{columns}
\end{frame}

\begin{frame}{Objetivos de Aprendizaje (OA) e Indicadores de Evaluación (IE)}
	\begin{columns}
		\begin{column}{.5\linewidth}
			\begin{block}{OA}
				Verbo + conocimiento
				\begin{itemize}
					\item Verbo en infinitivo.
					\item Verbo no observable.
				\end{itemize}
			\end{block}
			\begin{exampleblock}{Ejemplo}
				\begin{itemize}
					\item Conocer los tipos de evaluación según norma.
					\item Comprender los tipos de evaluación.
					\item Analizar la evaluación según validación y confiabilidad.
				\end{itemize}
			\end{exampleblock}
		\end{column}
		\begin{column}{.5\linewidth}
			\begin{block}{IE}
				Verbo + contenido + Condición
				\begin{itemize}
					\item Verbo medible.
					\item Verbo en indicativo presente y tercera persona singular o plural.
					\item Verbo igual o inferior al OA
					\item Contenido relacionado al OA.
					\item Condición observable: oral, escrito, gestual, gráfica
				\end{itemize}
			\end{block}
		\end{column}
	\end{columns}
\end{frame}

\begin{frame}{Algunas precisiones}

	\begin{itemize}[<+->]
		\item Los \emph{procesos cognitivos} son 6: Reconocer, Comprender, Aplicar, Analizar, Evaluar y Crear.
		\item cada \emph{proceso cognitivo} posee diferentes \textit{habilidades}, por ejemplo:
		      \begin{itemize}
			      \item Reconocer: Adquieren, Anotan, Citan (textualmente), Dicen, Definen, etc.
			      \item Comprender: Ejemplifican, Explican, Exponen, Identifican, Infieren, etc.
			      \item Aplicar:  Actúan, Ejecutan, Operan, Proceden, etc.
			      \item Analizar: Diferencian, Discriminan, Detectan, Especifican, Examinan, etc.
			      \item Evaluar: Argumentan, Categorizan, Conceptúan, Critican, Enjuician, Eligen, etc.
			      \item Crear: Agrupan, Clasifican, Explican, Establecen, Confeccionan, etc.
		      \end{itemize}
		\item Serán estas \textit{habilidades} las que usaremos para construir los IE, pero será el \emph{proceso cognitivo} al que pertenece el que índicaremos en nuestra \textit{tabla de especificaciones}.
	\end{itemize}
\end{frame}

\section{Tabla de especificaciones}

\begin{frame}{Tabla de especificaciones \cite{CASTILLO2010}}
	Finalmente, los OA e IE deben ubicarse en una tabla de especificaciones. ¿Qué es una tabla de especificaciones (TE)?
	\begin{center}
		\visible<2>{
			\begin{tabular}{|c|c|c|c|c|}
				\hline
				\textbf{OA}                 & \textbf{IE}       & \textbf{Habilidad del IE} & \textbf{N de preguntas} & \textbf{Dif.} \\
				\hline
				\multirow{3}{*}{Texto OA 1} & IE1.1 Texto IE1.1 & Reconocer                 & 3                       & 3F            \\
				\cline{2-5}
				                            & IE1.2 Texto IE1.2 & Comprender                & 3                       & 2F,1M         \\
				\cline{2-5}
				                            & IE1.3 Texto IE1.3 & Aplicar                   & 1                       & 1D            \\
				\hline
			\end{tabular}
		}
		
	\end{center}
\end{frame}

\begin{frame}{Prueba 2}
	Usted deberá:
	\begin{itemize}[<+->]
		\item Crear \textbf{un} \textit{Objetivo de Aprendizaje} de alto nivel cognitivo (Analizar, Evaluar o Crear) (1 pto.)
		\item Ubicarlo dentro de la tabla taxonómica \cite{KRATWOHL2002} (4 pts.)
		\item Crear al OA anterior \textbf{tres} \textit{indicadores de evaluación} y denominarlos IE1.1, IE1.2 y IE1.3 (12 pts.)
		\item Ahora, crear \textbf{una} \textit{tabla de especificaciones} para el OA y los IE. (4).
		\item Por grupo se, debe enviar \textbf{una} tarea vía \textbf{Canvas} a más tardar el día \textbf{Viernes}. (2 pts)
	\end{itemize}
\end{frame}

\section{Bibliografía}

\begin{frame}[allowframebreaks]{Bibliografía}

	\bibliography{../bibliografia}

\end{frame}

{
\setbeamertemplate{navigation symbols}{}
\begin{frame}
	\makebox[\linewidth]{\includegraphics[width=\paperwidth]{../template/background_last_page}}
\end{frame}
}

\end{document}
