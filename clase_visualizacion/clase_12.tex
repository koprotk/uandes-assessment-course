\documentclass[11pt, aspectratio=169, xcolor=table]{beamer} %aspectratio=169 presentación en 16:9; xcolor=table usar color en las tablas
\usepackage[utf8]{inputenc}
\usepackage[T1]{fontenc}
\usepackage{lmodern}
\usepackage[spanish]{babel}
%-------------Beamer------------------%
 \usetheme{Cuerna} 
 \usecolortheme{bluesimplex} %color del paquete
%-------------Marca de agua ---------------- (quitar comentario si se desea activar)
%\usepackage{background}
%\backgroundsetup{
%	placement=center,
%	scale=4,
%	contents={Preliminar},
%	opacity=1
%}
%\setbeamertemplate{background}{\BgMaterial}
%--------------------------------------------

\usepackage{graphicx} %incluír imágenes
\usepackage{hyperref} %agregar urls
\usepackage{ragged2e} %alinear a la izquierda y justificar.
% \setbeamertemplate{bibliography item}[text] Colocar números en lugar de íconos.
\bibliographystyle{../apalike-es} %archivo de estilo apa ../apa-es en español en el directorio local, otros estilos: plain.
\usepackage{pgfplots}
\usepackage{pgf-pie}
\usepackage{wasysym} %agregar un cuadrado como símbolo de lista
\usepackage{tabularx} % para ajuste del tamaño de la tabla
\usepackage{multirow} % juntar filas en una tabla

\newcommand\encircle[1]{% Poder crear letras dentro de círculos
	\tikz[baseline=(X.base)] 
	\node (X) [draw, shape=circle, inner sep=0] {\strut #1};}

\newcommand{\comment}[1]{} % crear comentario multilinea

\begin{document}
	\author{Prof. Daniel Muñoz \\ 
		\href{mailto:dmunoz@miuandes.cl}{\texttt{dmunoz@miuandes.cl}}}
	\title{Visualización de datos de evaluaciones sumativas}
	%\subtitle{subtitulo}
	\logo{
		\includegraphics[width=1.8cm]{../marca_principal.png}
	}
	\institute{Facultad de Educación, Universidad de los Andes}
	\date{\today{}} %agregar el día de hoy
	%\subject{subject}
	%\setbeamercovered{transparent}
	%\setbeamertemplate{navigation symbols}{}
	\begin{frame}[plain]
		\setbeamertemplate{logo}{} %eliminar logo de la portada
		\maketitle
	\end{frame}

	\section{Repaso de la clase anterior}
	
		\begin{frame}{El ciclo de la evaluación formativa. \cite{AC2010}}
		\begin{columns}
			\begin{column}{.5\linewidth}
				\begin{itemize}
					\item<1-> El ciclo de la evaluación formativa.
					\item<2-> Consiste de tres etapas:
					\begin{enumerate}
						\item<3-> ¿Hacia donde voy?
						\item<4-> ¿Dónde estoy?
						\item<5-> ¿Cómo sigo?
					\end{enumerate}
					\item<6-> En esto la \textit{retroalimentación} es fundamental ya que nos permite superar esa \textit{brecha}.
				\end{itemize}
			\end{column}
			\begin{column}{.5\linewidth}
				\only<1-2>{\includegraphics[width=\linewidth]{../imagenes/ciclo_formative}}
				\only<3>{Los objetivos de aprendizaje, criterios de logro deben ser de explicito conocimiento y entendimiento. El estudiante debe saber \textit{Para qué vino a clases}}\\
				\only<4>{El estudiante debe saber cual es su nivel, dónde está en al ruta del aprendizaje, para eso evaluamos y le señalamos exactamente cómo está el aprendizaje logrado hasta ahora  y ...}
				\only<5>{Cómo mejorar para llegar al destino}
				\only<6>{
					\begin{tikzpicture}[node distance= 3cm]
						\node[rectangle, rounded corners, minimum width=3cm, minimum height=1cm, text centered, draw=black] (zda) {Estado actual};
						\node[rectangle, rounded corners, minimum width=3cm, minimum height=1cm, text centered, draw=black, below of = zda] (zdp) {Aprendizaje};
						\draw[|-|,thick,>=stealth] ([xshift = 2cm] zda.south)  -- node[anchor=west] {Brecha} ([xshift = 2cm] zdp.north);
						\draw[>-<,thick,>=stealth] (zda) -- node[anchor=east] {Retroalimentación} (zdp);
				\end{tikzpicture}}
			\end{column}
		\end{columns}
	\end{frame}
	
	\section{La retroalimentación efectiva}
	
	\begin{frame}{La retroalimentación efectiva}
		\begin{columns}
			\begin{column}{.5\linewidth}
				\begin{itemize}[<+->]
					\item Primero que todo, ¿vale la pena retroalimentar?
					\item ¿En qué consiste una retroalimentación?
					\item Una correcta retroalimentación debe responder a criterios de calidad.
				\end{itemize}
			\end{column}
			\begin{column}{.5\linewidth}
				\begin{figure}
					\includegraphics<1>[width=\linewidth]{../imagenes/feedback_hattie}
					\only<2>{
						\begin{columns}
							\begin{column}{.3\linewidth}
								\begin{block}{}
									\flushright
									Lo que el estudiante hizo bien
								\end{block}	
							\end{column}
							\begin{column}{.3\linewidth}
								\centering +
							\end{column}
							\begin{column}{.3\linewidth}
								\begin{block}{}
									\flushleft Como mejorar
								\end{block}	
							\end{column}
						\end{columns}
					}
					\only<1>{\caption{``Hattie Rank''  \cite{HATTIE2017}} }
					\only<2>{\caption{Estructura de una \textit{retroalimentación}}}
				\end{figure}
				\only<3->{
					Una \textit{retroalimentación efectiva} debe:
					\begin{itemize}
						\item<4-> estar relacionada con la meta.
						\item<5-> entregar pistas, \textbf{no soluciones}.
						\item<6-> captar el razonamiento del estudiante.
						\item<7-> ser usada por el estudiante.
						\item<8-> enfocarse en el aprendizaje y \textbf{no} el individuo.
					\end{itemize}
				}
			\end{column}
		\end{columns}
	\end{frame}
	
	\section{De la data a la visualización}
	
	\begin{frame}{Visualización de datos: estética}
		\begin{columns}
			\begin{column}{.5\linewidth}
				\begin{itemize}[<+->]
					\item Se pueden ver tres gráfico para A = 3, B = 5 y C = 4
					\item El gráfico \textit{ugly} posee colores diferentes para A, B y C muy brillantes los cuales no aportan nada, tipografías diferentes y un fondo muy marcado.
					\item El gráfico \textit{bad} las barras no permiten hacer comparaciones, presenta escalas diferentes.
					\item El gráfico \textit{wrong} al no tener un eje Y no podemos saber nada respecto del gráfico ni establecer comparaciones.
				\end{itemize}
			\end{column}
			\begin{column}{.5\linewidth}
				\includegraphics[width=\linewidth]{grafico_1}
			\end{column}
		\end{columns}
	\end{frame}
	
	\section{Sistemas de coordenadas y ejes}
	\begin{frame}{Sistemas de coordenadas y ejes}
		\begin{columns}
			\begin{column}{.5\linewidth}
				Existen diversos tipos de coordenadas y escalas
				\begin{itemize}[<+->]
					\item Cartesianas clásicas
					\item Cartesiana con etiquetas ¿Por qué no unir con líneas?
					\item Tipos de ejes en escalas: lineales y logarítmicas.
					\item En general escalas logarítmicas permiten acercar valores que están muy alejados, no es de uso en educación.
				\end{itemize}
			\end{column}
			\begin{column}{.5\linewidth}
				\includegraphics<1>[width=\linewidth]{cartesiana_clasica}
				\only<2>{
				\begin{tikzpicture}
					\begin{axis}
						[
						,width=7cm
						,xlabel=Mes
						,ylabel=Procentaje de Logro
						,xtick=data,
						%,xtick={0,1,...,3}
						,xticklabels={Mar,Abr,May,Junio}
						]
						\addplot+[only  marks] coordinates
						{(0,18.26) (1,21.47) (2,24.58) (3,24.95)};
					\end{axis}
				\end{tikzpicture}
				}
				\includegraphics<3>[width=\linewidth]{ejes}
				\includegraphics<4>[width=\linewidth]{escala_logaritmica}
			\end{column}	
		\end{columns}
	\end{frame}

	\begin{frame}{Escalas de color}
		\begin{columns}
			\begin{column}{.5\linewidth}
				Las escalas son importantes, es importante usar escalas de colores llamativos, pero no excesivos, ejemplo de escalas de color:
				\begin{itemize}[<+->]
					\item Okabe Ito (2008)
					\item ColorBrewer Dark2 (Brewer, 2017)
					\item ggplot2 hue
					\item Un buen uso del color permite centrar la atención
				\end{itemize}
			\end{column}
			\begin{column}{.5\linewidth}
				\includegraphics<1>[width=\linewidth]{okabeito}
				\includegraphics<2>[width=\linewidth]{brewer}
				\includegraphics<3>[width=\linewidth]{ggplot2}
				\includegraphics<4>[width=\linewidth]{color_plot}
			\end{column}
		\end{columns}	
	\end{frame}
	
	\begin{frame}{Gráficos para representar cantidades}
		\begin{columns}
			\begin{column}{.5\linewidth}
				\begin{itemize}[<+->]
					\item barras verticales
					\item barras verticales agrupadas
					\item barras horizontales
					\item barras horizontales agrupadas
					\item barras verticales apiladas
					\item barras horizontales apiladas
					\item mapas de calor
				\end{itemize}
			\end{column}
			\begin{column}{.5\linewidth}
				\includegraphics<1>[width=\linewidth]{barrasv}
				\includegraphics<2>[width=\linewidth]{barrasva}
				\includegraphics<3>[width=\linewidth]{barrash}
				\includegraphics<4>[width=\linewidth]{barrasha}
				\includegraphics<5>[width=\linewidth]{barrasvs}
				\includegraphics<6>[width=\linewidth]{barrashs}
				\includegraphics<7>[width=.8\linewidth]{mapacalor}
			\end{column}
		\end{columns}
	\end{frame}
	
	\begin{frame}{Algunos gráficos de distribución}
		\begin{columns}
			\begin{column}{.5\linewidth}
				\begin{itemize}[<+->]
					\item histograma (variable discreta)
					\item gráfico densidad (variable continua)
				\end{itemize}
			\end{column}
			\begin{column}{.5\linewidth}
				\includegraphics<1>[width=\linewidth]{histograma}
				\includegraphics<2>[width=\linewidth]{density}
			\end{column}
		\end{columns}
	\end{frame}
	
	\begin{frame}{Gráficos de proporciones}
		\begin{columns}
			\begin{column}{.5\linewidth}
				\begin{itemize}[<+->]
					\item Gráficos de torta
					\item Barras
				\end{itemize}
			\end{column}
			\begin{column}{.5\linewidth}
				\only<1>{
					\begin{tikzpicture}
							\pie[radius=2]{
								88/Acuerdo,
								12/Desacuerdo}
					\end{tikzpicture}}
				\includegraphics<2>[width=\linewidth]{barras1}
				\includegraphics<2>[width=\linewidth]{barras2}
			\end{column}
		\end{columns}
	\end{frame}
	
	\begin{frame}{Gráficos xy}
		\begin{columns}
			\begin{column}{.5\linewidth}
				\begin{itemize}[<+->]
					\item Dispersión
					\item Burbujas
					\item Dispersión con tendencia
				\end{itemize}
			\end{column}
			\begin{column}{.5\linewidth}
				\includegraphics<1>[width=\linewidth]{dispersion}
				\includegraphics<2>[width=\linewidth]{burbuja}
				\includegraphics<3>[width=\linewidth]{despersion_t}
			\end{column}
		\end{columns}
	\end{frame}
	
	\begin{frame}{Principio de proporcionalidad de tinta}
		\begin{columns}
			\begin{column}{.5\linewidth}
				\begin{itemize}[<+->]
					\item ``Cuando una región es ennegrecida (colorida) para representar un valor numérico, el área ennegrecida debe ser proporcional al valor correspondiente'' (Bergstrom and West, 2016)
					\item Este gráfico tiene problemas ¿cuales?
					\item Mejor así
				\end{itemize}
			\end{column}
			\begin{column}{.5\linewidth}
				\includegraphics<2>[width=\linewidth]{propink_no}
				\includegraphics<3>[width=\linewidth]{propink_yes}
			\end{column}
		\end{columns}
	\end{frame}
	
	\section{Títulos, leyendas y tablas}
	
	\begin{frame}{Leyendas}
		\begin{columns}
			\begin{column}{.5\linewidth}
				\begin{itemize}[<+->]
					\item Ojo con el orden de las entradas en la leyenda
					\item Mejor así
					\item Sin leyenda, a veces mejor.
				\end{itemize}
			\end{column}
			\begin{column}{.5\linewidth}
				\includegraphics<1>[width=\linewidth]{leyend_bad}
				\includegraphics<2>[width=\linewidth]{leyend_good}
				\includegraphics<3>[width=\linewidth]{leyend_better}
			\end{column}
		\end{columns}
	\end{frame}

	\begin{frame}{Tablas}
		\begin{columns}
			\begin{column}{.5\linewidth}
				\begin{itemize}[<+->]
					\item Cuidado con colores y diseño
				\end{itemize}
			\end{column}
			\begin{column}{.5\linewidth}
				\includegraphics[width=\linewidth]{tablas}
			\end{column}
		\end{columns}
	\end{frame}
	
	\begin{frame}{Evita las líneas}
		\begin{columns}
			\begin{column}{.5\linewidth}
				\begin{itemize}[<+->]
					\item Siempre es mejor evitar las líneas
					\item mejor así
				\end{itemize}
			\end{column}
			\begin{column}{.5\linewidth}
				\only<1>{
				\begin{tikzpicture}
					\begin{axis}[width=\linewidth]
						\addplot [
						ybar,
						] coordinates {
							(0,3) (1,2) (2,4) (3,1) (4,2)
						};
					\end{axis}
				\end{tikzpicture}
				}
				\only<2>{
				\begin{tikzpicture}
					\begin{axis}[width=\linewidth]
						\addplot [
						ybar,fill=blue
						] coordinates {
							(0,3) (1,2) (2,4) (3,1) (4,2)
						};
					\end{axis}
				\end{tikzpicture}
				}
			\end{column}
		\end{columns}
	\end{frame}
	
	\begin{frame}{Definitivamente evite los 3D}
		\begin{columns}
			\begin{column}{.5\linewidth}
				\begin{itemize}[<+->]
					\item Si tiene un gráfico en 3D trate...
					\item de dejarlo en 2D siempre que pueda.
				\end{itemize}
			\end{column}
			\begin{column}{.5\linewidth}
				\includegraphics<1>[width=\linewidth]{3D_no}
				\includegraphics<2>[width=\linewidth]{3D_yes}
			\end{column}
		\end{columns}
	\end{frame}
	
	\section{Bibliografía}
	
	\begin{frame}[allowframebreaks]{Bibliografía}
		\nocite{WILKE2019}
		
		\bibliography{../bibliografia}
		
	\end{frame}
	
\end{document}