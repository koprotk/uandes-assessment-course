\documentclass[11pt, aspectratio=169, xcolor=table,hyphens]{beamer}
\usepackage[utf8]{inputenc}
\usepackage[T1]{fontenc}
\usepackage{lmodern}
\usepackage[spanish]{babel}
\usetheme{default}
\usepackage{pdfrender}
\usepackage{tikz}
\usepackage{graphicx}
\usepackage{xcolor}
\usepackage{hyperref}
\usepackage{cite} %permite salto de línea en referencias largas
\usepackage{ragged2e}
\usepackage{smartdiagram}
\bibliographystyle{../apalike-es} %archivo de estilo apa ../apa-es en español en el directorio
\usepackage{pgfplots}
\usepackage[]{wasysym}
\usepackage{tabularx} % para ajuste del tamaño de la tabla
\usepackage{multirow} % juntar filas en una tabla

\usepackage{../template/uandes169template}

\begin{document}
\author{Prof. Daniel Muñoz \\
	\href{mailto:dmunoz@miuandes.cl}{\texttt{dmunoz@miuandes.cl}}}
\title{Tabla de especificaciones}
\subtitle{Clase 7}
\date{23 de abril de 2025} %agregar el día de hoy
%\subject{subject}
%\setbeamercovered{transparent}
%\setbeamertemplate{navigation symbols}{}
\maketitle

\section{Resumen}

\begin{frame}{Resumen: Módulo 2}
	\begin{columns}
		\begin{column}{.5\linewidth}
			\only<1>{\begin{block}{Taxonomía \cite{STATEU2021}}
					Recurso pedagógica que orienta a los equipos tener un lenguaje único sin ambigüedades
				\end{block}}
			\only<2>{\begin{block}{Taxonomía de Bloom \cite{BLOOM1956} }
					Primera taxonomía educativa originada por Benjamin Bloom (1913 - 1999), era una clasificación de habilidades (verbos) agrupadas en Dominios Cognitivos, del más elemental Conocimiento hasta la más compleja Evaluar.
				\end{block}}
			\only<3>{\begin{block}{Taxonomía Revisada de Anderson \cite{KRATWOHL2002} }
					En esta taxonomía se hace una revisión de la taxonomía de Bloom, convirtiéndola de unidimensional a bidimensional.
				\end{block}}
		\end{column}
		\begin{column}{.5\linewidth}
			\only<1>{
				\begin{figure}
					\includegraphics[width=.8\linewidth]{../imagenes/same_language}
					\caption{Siempre es bueno tener una taxonomía que oriente lo que hacemos}
				\end{figure}}
			\only<2>{
				\begin{block}{Taxonomía de Bloom}
					\begin{enumerate}
						\item Conocer
						\item Comprender
						\item Aplicar
						\item Analizar
						\item Sintetizar
						\item Evaluar
					\end{enumerate}
				\end{block}}
			\only<3>{
				\begin{block}{Taxonomía de Anderson \& Kratwohl}
					\begin{tikzpicture}[node distance= 2.2cm]
						\node[rectangle, fill=gray!50] (org) {Original};
						\node[rectangle, rounded corners, minimum width=3cm, minimum height=1cm, text centered, draw=black, below of = org] (knowB) {Conocer};
						\node[rectangle, rounded corners, minimum width=3cm, minimum height=1cm, text centered, draw=black, below right of= knowB] (knowK) {Conocer};
						\node[rectangle, rounded corners, minimum width=3cm, minimum height=1cm, text centered, draw=black, below left of= knowB] (rem) {Recordar};
						\node[rectangle, fill=gray!50, below right of = rem] (rev ){Revisada};
						\draw[->,thick,>=stealth] (knowB)  -- node[anchor=west] {Conocimiento} (knowK);
						\draw[->,thick,>=stealth] (knowB) -- node[anchor=east] {Proceso Cognitivo} (rem);
					\end{tikzpicture}
				\end{block}}
		\end{column}
	\end{columns}
\end{frame}

\begin{frame}{Objetivos de Aprendizaje (OA) e Indicadores de Evaluación (IE)}
	\begin{columns}
		\begin{column}{.5\linewidth}
			\begin{block}{OA}
				Verbo + conocimiento
				\begin{itemize}
					\item Verbo en infinitivo.
					\item Verbo no observable.
				\end{itemize}
			\end{block}
			\begin{exampleblock}{Ejemplo}
				\begin{itemize}
					\item Conocer los tipos de evaluación según norma.
					\item Comprender los tipos de evaluación.
					\item Analizar la evaluación según validación y confiabilidad.
				\end{itemize}
			\end{exampleblock}
		\end{column}
		\begin{column}{.5\linewidth}
			\begin{block}{IE}
				Verbo + contenido + Condición
				\begin{itemize}
					\item Verbo medible.
					\item Verbo en indicativo presente y tercera persona singular o plural.
					\item Verbo igual o inferior al OA
					\item Contenido relacionado al OA.
					\item Condición observable: oral, escrito, gestual, gráfica
				\end{itemize}
			\end{block}
		\end{column}
	\end{columns}
\end{frame}

\begin{frame}{Cuidado con...}
	\begin{columns}
		\begin{column}{.5\linewidth}
			\begin{exampleblock}{Lea los siguientes IE.}
				\begin{enumerate}
					\item Comprende el funcionamiento del sistema digestivo mediante una maqueta.
					\item Desempeña infraestructura tecnológica manualmente.
					\item Los alumnos reconocen ser felices gestualmente.
					\item Diferencia para reconocer una argumentación oral.
				\end{enumerate}
			\end{exampleblock}
		\end{column}
		\begin{column}{.5\linewidth}
			\begin{itemize}
				\item<2-> En 1 no puedes ver un funcionamiento con una maqueta.
				\item<3-> Desempeñar es \textit{Ejercer las obligaciones inherentes a una profesión, cargo u oficio.}\cite{RAE2021}
				\item<4-> Nunca colocar \textit{los alumnos} ya que siempre esta dirigido a lo que ellos hacen, por tanto es tremendamente redundante.
				\item<5-> Dos verbos
			\end{itemize}

		\end{column}
	\end{columns}
\end{frame}

\section{Tabla de especificaciones}

\begin{frame}{Tabla de especificaciones \cite{CASTILLO2010}}
	Finalmente, los OA e IE deben ubicarse en una tabla de especificaciones. ¿Qué es una tabla de especificaciones (TE)?
	\begin{center}
		\visible<2>{
				% Please add the following required packages to your document preamble:
				% \usepackage{multirow}
				\begin{table}[]
					\begin{tabular}{|l|l|l|l|l|l|}
						\hline
						OA                   & C                  & IE  & H  & C  & N \\ \hline
						\multirow{3}{*}{OA1} & \multirow{3}{*}{C} & IE1 & H1 & C1 & 5  \\ \cline{3-6}
						                     &                    & IE2 & H2 & C2 & 3  \\ \cline{3-6}
						                     &                    & IE3 & H3 & C3 & 1  \\ \hline
					\end{tabular}
				\end{table}
		}
	\end{center}
\end{frame}

\begin{frame}{Analicemos algunos elementos}
	\begin{block}{Observe que:}
		\begin{itemize}
			\item Llamaremos habilidad al proceso cognitivo del IE.
			\item Los IE de evaluación están ordenados de la habilidad más simple a la más compleja.
			\item El último indicador es exactamente igual al OA en habilidad y contenido (temática y tipo).
			\item Los indicadores logran \textbf{tributar} a la totalidad del OA y juntos forman una \textbf{unidad de sentido}.
		\end{itemize}
	\end{block}
	\begin{block}{Clasificación}
		\begin{tabularx}{\linewidth}{|X|X|}
			\hline
			Reconocer, Comprender, Aplicar & Bajo nivel cognitivo \\
			\hline
			Analizar, Evaluar, Crear       & Alto nivel cognitivo \\
			\hline
		\end{tabularx}
	\end{block}
\end{frame}

\section{Bibliografía}

\begin{frame}[allowframebreaks]{Bibliografía}

	\bibliography{../bibliografia}

\end{frame}

{
\setbeamertemplate{navigation symbols}{}
\begin{frame}
	\makebox[\linewidth]{\includegraphics[width=\paperwidth]{../template/background-last-page}}
\end{frame}
}

\end{document}
